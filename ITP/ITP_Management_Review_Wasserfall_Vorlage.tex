% David Koch

\documentclass[
	headings=optiontotocandhead,% Erweiterung für das optionale Argument der
	% Gliederungsbefehle aktiviert.
	oneside,
	numbers=noenddot,% Keine Punkte am Ende der Gliederungsnummern und davon
	% abgeleiteten Nummern
	toc=flat, %Flache TOC --- kann man anpassen (auskommentieren)
	10pt, % Schriftgröße
	parskip=full, % Abstand zwischen Absätzen (ganze Zeile)
	listof=totoc, % Verzeichnisse im Inhaltsverzeichnis aufführen
	listof=flat, % mehr Abstand für grosse Zahlen
	numbers=noenddot, % kein Punkt am Ende bei Nummern
	%%enlargefirstpage,% Gibt es bei scrartcl nicht!!!!
	bibliography=totoc, % Literaturverzeichnis im Inhaltsverzeichnis aufführen
	%index=totoc, % Index im Inhaltsverzeichnis aufführen
	%captions=tableheading, % Beschriftung von Tabellen für Ausgabe oberhalb
	% der Tabelle formatieren
	%draft % Status des Dokuments (final/draft) draft hinzufügen zum anziegen
	%%der zeilen ende
	a4paper,DIV=14,
	% captions=tablesignature,
]{scrartcl}

\setcounter{secnumdepth}{3}

\usepackage[T1]{fontenc}
\usepackage[utf8]{inputenc}

\usepackage[english, ngerman]{babel, varioref} % your native language must be the last one!!

\usepackage{lastpage}
\usepackage{listings}
\usepackage{blindtext}

%% Aufzählungen nicht so weit einrücken
\usepackage[inline]{enumitem}
%\setitemize{leftmargin=*}
% Listen etwas wenige einrücken, erfordert enumitem
\setitemize{labelindent=2em,labelsep=0.5cm,leftmargin=12ex}

\usepackage{lmodern}

\usepackage{xspace}

\usepackage{graphicx}
\graphicspath{ {./} }

%%? \usepackage{textcomp}
\usepackage[hyphens]{url}
\usepackage{makeidx}
\makeindex
%%? \usepackage{graphicx}
\usepackage[numbers]{natbib}
\PassOptionsToPackage{normalem}{ulem}
\usepackage{ulem}

\usepackage{needspace}

\setlength\partopsep{0.5ex}%schoenere Listen
\usepackage[bottom]{footmisc}%fussnote ganz unten

\usepackage[]{microtype}
\UseMicrotypeSet[protrusion]{basicmath} % disable protrusion for tt fonts

\usepackage{multirow}   % Allows table elements to span several rows.
\usepackage{booktabs}   % Improves the typesettings of tables.
\usepackage{subcaption} % Allows the use of subfigures and enables their referencing.
\usepackage[ruled,linesnumbered]{algorithm2e} % Enables the writing of pseudo code.
\usepackage[usenames,dvipsnames,table]{xcolor} % Allows the definition and use of colors. This package has to be included before tikz.
\usepackage{nag}       % Issues warnings when best practices in writing LaTeX documents are violated.
\usepackage{todonotes} % Provides tooltip-like todo notes.

\usepackage{color}
\usepackage[binary-units]{siunitx}

%% Override default figure placement To be within the flow of the text rather
%% than on it's own page.
% \usepackage{float}
% \makeatletter
% \def\fps@figure{H}
% \makeatother

%% bei vielen Bildern o.ä sinnvoll: Seite muss nicht bis ganz unten gefüllt werden
% \raggedbottom

%\usepackage{footbib} %  footcite, needs other tooling
%% for pandoc2 images
\makeatletter
\def\maxwidth{\ifdim\Gin@nat@width>\linewidth\linewidth\else\Gin@nat@width\fi}
\def\maxheight{\ifdim\Gin@nat@height>\textheight\textheight\else\Gin@nat@height\fi}
\makeatother
% Scale images if necessary, so that they will not overflow the page
% margins by default, and it is still possible to overwrite the defaults
% using explicit options in \includegraphics[width, height, ...]{}
\setkeys{Gin}{width=\maxwidth,height=\maxheight,keepaspectratio}

%% bessere Suche im PDF
\input{glyphtounicode}
\pdfgentounicode=1
%%%%%%%%%%%%%%%%%%%%%%%%%%%%%%%%%%%%%%%%%%%%%%%%%%%%%%%%%%%%%%%%%%%%%%%%%%%%%%%%%%

%  Kopf und Fußzeilen -- links und rechts verschieden
\newcommand{\kopfbild}{\voffset7mm\includegraphics[width=25mm]{HTL3RLogo}}
\newcommand{\kopfHTL}{\sffamily{\textbf{\large{Projekthandbuch HTL3R}}}}

\usepackage[automark,footsepline,plainfootsepline]{scrlayer-scrpage}
\setkomafont{pageheadfoot}{\normalcolor\footnotesize\scshape}
\setkomafont{pagenumber}{\normalfont\normalsize}
\clearpairofpagestyles
\ihead{\headmark}
\ihead{\kopfbild}
\ohead{\kopfHTL}
\ifoot{\smaller{Höhere Technische Bundeslehranstalt Wien 3 | Rennweg 89b | 1030 Wien | \textcolor{orange}{www.htl.rennweg.at}}}
\ofoot{Seite \pagemark/\pageref{LastPage}}
\ModifyLayer[addvoffset=-.6ex]{scrheadings.foot.above.line}% Linie verschieben
\ModifyLayer[addvoffset=-.6ex]{plain.scrheadings.foot.above.line}% Linie verschieben
\setlength{\headheight}{32pt}

% alle Seiten mit Kopfzeile
%\renewcommand{\chapterpagestyle}{scrheadings}

%% Code Beispiele
%% eine Variante
\usepackage{listings}
\renewcommand{\lstlistingname}{\inputencoding{utf8}Listing}

\usepackage{tabularx}
\usepackage{scrhack}

\usepackage{array}
\newcommand\Tstrut{\rule{0pt}{3.2ex}}         % = `top' strut
\newcommand\Bstrut{\rule[-1.5ex]{0pt}{0pt}}   % = `bottom' strut

\newenvironment{nstabbing}
	{\setlength{\topsep}{-\parskip}
		\setlength{\partopsep}{-\parskip}
		\tabbing}
	{\endtabbing}

\usepackage{titlesec}
% \titleformat{?Überschriftenklasse?}[Absatzformatierung?]{?Textformatierung?} {?Nummerierung?}{?Abstand zwischen Nummerierung und Überschriftentext?}{?Code vor der Überschrift?}[?Code nach der Überschrift?]
\titleformat{\section}[hang]{\Large\bfseries\sffamily}{\thesection\quad}{-1.2ex}{}
\titleformat{\subsection}[hang]{\large\bfseries\sffamily}{\thesubsection\quad}{-1.2ex}{}
\titleformat{\subsubsection}[hang]{\large\bfseries\sffamily}{\thesubsubsection\quad}{-1.2ex}{}
\titleformat{\paragraph}[hang]{\large\bfseries\sffamily}{\theparagraph\quad}{-1.2ex}{}

% \titlespacing{?Überschriftenklasse?}{?Linker Einzug?}{?Platz oberhalb?}{?Platz unterhalb?}[?rechter Einzug?]
\titlespacing{\section}{0pt}{6pt}{6pt}
\titlespacing{\subsection}{0pt}{6pt}{0pt}
\titlespacing{\subsubsection}{0pt}{6pt}{0pt}
\titlespacing{\paragraph}{0pt}{6pt}{0pt}

%% sollte das letzte Package sein
\usepackage[unicode=true,
bookmarks=true,bookmarksnumbered=false,bookmarksopen=false,
breaklinks=true,pdfborder={0 0 0},backref=false,colorlinks=false]
{hyperref}
\hypersetup{pdftitle={ITP-Vorlage},
	pdfauthor={Wer auch immer},
	pdfsubject={ITP},
	pdfkeywords={4CN, ITP}}
\urlstyle{same} % don't use monospace font for urls

% Auch Fußnoten bündig ausrichten
\deffootnote[]{1em}{1em}{\textsuperscript{\thefootnotemark\ }}
%% setup
\sloppy % weniger Meldungen
\voffset7mm % etwas nach unten

%%%%%%%%%%%%%%%%%%%%%%%%%%%%%%%%%%%%%%%%%%%%%%%%%%%%%%%%%%%%%%%%%%%%%%%%%%%%%%%%%%
\begin{document}
%% schöner: 10000 -- gar keine, 1000 als Mittelweg
\clubpenalty = 1000 % Schusterjungen verhindern
\widowpenalty = 1000 % Hurenkinder verhindern
\displaywidowpenalty = 1000

{\sffamily{\textbf{\LARGE{\textcolor{orange}{Dokumenttitel}}}}}\\
\noindent\rule{\textwidth}{0.1pt}
\begin{nstabbing}
	\hspace{4cm}\=\hspace{4cm}\=\hspace{4cm}\=\kill
	Projekttitel: \> \textbf{Projekttitel eingeben}\\
	Auftraggeber*in: \> \textbf{Vorname Nachname / Unternehmen (eine Person)}\\
	Auftragnehmer*in: \> \textbf{Vorname Nachname Projektleiter*in / Unternehmen}\\
	Schuljahr: \> \textbf{202X/2X}
	\> Klasse: \> \textbf{XXX}\\
\end{nstabbing}
{\smaller
	\begin{tabularx}{\textwidth}{l l l l}
	\hline
	\textbf{Version} & \textbf{Datum} & \textbf{Autorin/Autor} & \textbf{Änderung}\Tstrut  \\
	v1.0 & 05.09.2024 & Vorname Nachname & Beschreibung der Änderung, z.B. Erstellung...\Bstrut \\
	\hline
	\end{tabularx}
}

\section{Projektstatus}
\textbf{Berichtszeitraum: xx.xx.2024 bis xx.xx.2024} (bitte anpassen, normal 2 Wochen)

Kurze verbale Beschreibung, was wurde erreicht, welche Meilensteine, was nicht, welche Probleme. Wo steht das Projekt; 

Derzeitiger Ampelstatus: Grün/Gelb/Rot
\begin{enumerate}
	\item Grün: Termine, Ergebnisse, Ziele, Budget und Kosten sind im Plan
	\item Gelb: Es gibt leichte Abweichungen, die aber keinen Einfluss auf den Rahmen des Projektauftrags haben (max. 5\% Abweichung UND es ist möglich, die Abweichungen noch wettzumachen) => unten müssen dann auch die Maßnahmen beschrieben werden, wie man gegensteuern wird
	\item Rot: Die Inhalte des Projektauftrags (Ziele, Kosten, Termine) sind gefährdet (Abweichungen > 5\% oder es sind geringere Abweichungen, die aber bereits als fix betrachtet werden können) => unten müssen dann auch die Maßnahmen beschrieben werden, wie man gegensteuern wird
\end{enumerate}

\textbf{Termine}

{\smaller
	\begin{tabularx}{\textwidth}{|X|X|X|X|X|}
		\hline
		\textbf{\% abgeschlossen} & \textbf{Projektstatus} & \textbf{Erwartetes\newline Projektende} & \textbf{Geplantes Projektende} & \textbf{Abweichung} \\
		\hline
		- & - & - & - & - \\
		\hline
	\end{tabularx}
}
% \% abgeschlossen = Prozentwert, wie viel des Gesamtprojektes abgeschlossen ist. Ermittlung auf der Basis der abgeschlossenen Arbeiten (Stunden) im Verhältnis zu den noch zu erledigenden Arbeiten (in Stunden)

\textbf{Kosten}

{\smaller
	\begin{tabularx}{\textwidth}{|X|X|X|X|X|}
		\hline
		\textbf{Ist-Kosten} & \textbf{Restkosten} & \textbf{Erwartete\newline Gesamtkosten} & \textbf{Plankosten} & \textbf{Abweichung} \\
		\hline
		- & - & - & - & - \\
		\hline
	\end{tabularx}
}
% Ist-Kosten = bisher angefallene Kosten in € / bisher angefallene Stunden \\
% Restkosten = welche Kosten bzw. wieviel Stunden sind ab heute bis zum Projektende noch zu erwarten \\
% Erwartete Gesamtkosten = Ist-Kosten + Restkosten bzw. Ist-Aufwand (h) + Restaufwand (h) \\
% Plankosten = ursprünglich (Projektbeginn, Angebot) berechnete Kosten bzw. ursprünglich geschätzter Stundenaufwand \\
% Abweichung = Erwartete Gesamtkosten – Plankosten in € bzw. Erwarteter Gesamtaufwand – Planaufwand (h) \\

\textbf{Stunden}

{\smaller
	\begin{tabularx}{\textwidth}{|X|X|X|X|X|}
		\hline
		\textbf{Ist-Aufwand} & \textbf{Restaufwand} & \textbf{Erwarteter\newline Gesamtaufwand} & \textbf{Planaufwand} & \textbf{Abweichung} \\
		\hline
		- & - & - & - & - \\
		\hline
	\end{tabularx}
}
% Analog zu den Kosten (siehe oben)

\subsection{Teammotivation \colorbox{green!30}{:)} / \colorbox{yellow!30}{:|} / \colorbox{red!30}{:(}} % nicht zutreffende smileys löschen
Kurze Beschreibung der Motivation des Teams (gesamt, nicht individuell).

\section{Probleme im Projekt}
Kurze Schilderung von Problemen und deren Auswirkungen auf das Projekt.

\subsection{Problemlösungsstrategie}
Beschreibung der Strategie zur Bereinigung von diversen Problemen im Projekt. Bei gelbem und rotem Status unbedingt notwendig, sowie bei Angabe von Problemen im vorigen Punkt.

Welche Maßnahmen werden oder wurden getroffen, um das Projekt wieder in den grünen Bereich zu steuern?

\subsection{Handlungsbedarf seitens des Managements}
Sind Aktivitäten seitens des Managements (Projektauftraggeber, Projektbetreuung, Geschäftsleitung, Management) notwendig? Wenn ja, welche?

\begin{enumerate}
	\item Abnahme des Pflichtenhefts bis .......
	\item Entscheidung über Auswahl des Screendesigns bis .......
\end{enumerate}

\section{Erledigte Arbeiten (vollständig)}
\begin{table}[h]
	\begin{tabularx} {\textwidth} {
			|>{\hsize=.12\hsize}X
			|>{\hsize=.08\hsize}X
			|>{\hsize=.34\hsize}X
			|>{\hsize=.10\hsize}X
			|>{\hsize=.12\hsize}X
			|>{\hsize=.12\hsize}X
			|>{\hsize=.09\hsize}X|
		}
		
		\hline
		\rowcolor[HTML]{D9D9D9} 
		\textbf{\normalsize{Bearbeiter}} & \textbf{\normalsize{PSP-Code}} & {\textbf{\normalsize{Tätigkeit}}} & \textbf{\normalsize{Ort}} & \textbf{\normalsize{Dauer geplant (h)}} & \textbf{\normalsize{Dauer benötigt (h)}} & \textbf{\normalsize{Status}} \\ \hline
		MAI & 1.2.1 & Erstellen des Screendesigns & Schule & 1,5 & 2,5 & \cellcolor{green!30} \\ \hline
		- & - & - & - & - & - & \cellcolor{yellow!30} \\ \hline
		- & - & - & - & - & - & \cellcolor{red!30} \\ \hline
	\end{tabularx}
\end{table}
Alle erledigten Arbeiten sollen hier gelistet sein, d.h. man kann die erledigten Arbeiten einer älteren Management Review belassen und fügt unten die erledigten Arbeiten der vergangenen 3 Wochen an. Die Daten können ggf. auch aus einem Zeiterfassungssystem übernommen werden. 

\section{Anstehende Arbeiten}
\begin{table}[h]
	\begin{tabularx} {\textwidth} {
			|>{\hsize=.12\hsize}X
			|>{\hsize=.08\hsize}X
			|>{\hsize=.54\hsize}X
			|>{\hsize=.12\hsize}X
			|>{\hsize=.15\hsize}X|
		}
		
		\hline
		\rowcolor[HTML]{D9D9D9} 
		\textbf{\normalsize{Bearbeiter}} & \textbf{\normalsize{PSP-Code}} & {\textbf{\normalsize{Tätigkeit}}} & \textbf{\normalsize{Dauer geplant (h)}} & \textbf{\smaller{Fertigstellung geplant}} \\ \hline
		- & - & - & - & - \\ \hline
	\end{tabularx}
\end{table}

\end{document}