% David Koch

\documentclass[
	headings=optiontotocandhead,% Erweiterung für das optionale Argument der
	% Gliederungsbefehle aktiviert.
	oneside,
	numbers=noenddot,% Keine Punkte am Ende der Gliederungsnummern und davon
	% abgeleiteten Nummern
	toc=flat, %Flache TOC --- kann man anpassen (auskommentieren)
	10pt, % Schriftgröße
	parskip=full, % Abstand zwischen Absätzen (ganze Zeile)
	listof=totoc, % Verzeichnisse im Inhaltsverzeichnis aufführen
	listof=flat, % mehr Abstand für grosse Zahlen
	numbers=noenddot, % kein Punkt am Ende bei Nummern
	%%enlargefirstpage,% Gibt es bei scrartcl nicht!!!!
	bibliography=totoc, % Literaturverzeichnis im Inhaltsverzeichnis aufführen
	%index=totoc, % Index im Inhaltsverzeichnis aufführen
	%captions=tableheading, % Beschriftung von Tabellen für Ausgabe oberhalb
	% der Tabelle formatieren
	%draft % Status des Dokuments (final/draft) draft hinzufügen zum anziegen
	%%der zeilen ende
	a4paper,DIV=14,
	% captions=tablesignature,
]{scrartcl}

\setcounter{secnumdepth}{3}

\usepackage[T1]{fontenc}
\usepackage[utf8]{inputenc}

\usepackage[english, ngerman]{babel, varioref} % your native language must be the last one!!

\usepackage{lastpage}
\usepackage{listings}
\usepackage{blindtext}

%% Aufzählungen nicht so weit einrücken
\usepackage[inline]{enumitem}
%\setitemize{leftmargin=*}
% Listen etwas wenige einrücken, erfordert enumitem
\setitemize{labelindent=2em,labelsep=0.5cm,leftmargin=12ex}

\usepackage{lmodern}

\usepackage{xspace}

\usepackage{graphicx}
\graphicspath{ {./} }

%%? \usepackage{textcomp}
\usepackage[hyphens]{url}
\usepackage{makeidx}
\makeindex
%%? \usepackage{graphicx}
\usepackage[numbers]{natbib}
\PassOptionsToPackage{normalem}{ulem}
\usepackage{ulem}

\usepackage{needspace}

\setlength\partopsep{0.5ex}%schoenere Listen
\usepackage[bottom]{footmisc}%fussnote ganz unten

\usepackage[]{microtype}
\UseMicrotypeSet[protrusion]{basicmath} % disable protrusion for tt fonts

\usepackage{multirow}   % Allows table elements to span several rows.
\usepackage{booktabs}   % Improves the typesettings of tables.
\usepackage{subcaption} % Allows the use of subfigures and enables their referencing.
\usepackage[ruled,linesnumbered]{algorithm2e} % Enables the writing of pseudo code.
\usepackage[usenames,dvipsnames,table]{xcolor} % Allows the definition and use of colors. This package has to be included before tikz.
\usepackage{nag}       % Issues warnings when best practices in writing LaTeX documents are violated.
\usepackage{todonotes} % Provides tooltip-like todo notes.

\usepackage{color}
\usepackage[binary-units]{siunitx}

%% Override default figure placement To be within the flow of the text rather
%% than on it's own page.
% \usepackage{float}
% \makeatletter
% \def\fps@figure{H}
% \makeatother

%% bei vielen Bildern o.ä sinnvoll: Seite muss nicht bis ganz unten gefüllt werden
% \raggedbottom

%\usepackage{footbib} %  footcite, needs other tooling
%% for pandoc2 images
\makeatletter
\def\maxwidth{\ifdim\Gin@nat@width>\linewidth\linewidth\else\Gin@nat@width\fi}
\def\maxheight{\ifdim\Gin@nat@height>\textheight\textheight\else\Gin@nat@height\fi}
\makeatother
% Scale images if necessary, so that they will not overflow the page
% margins by default, and it is still possible to overwrite the defaults
% using explicit options in \includegraphics[width, height, ...]{}
\setkeys{Gin}{width=\maxwidth,height=\maxheight,keepaspectratio}

%% bessere Suche im PDF
\input{glyphtounicode}
\pdfgentounicode=1
%%%%%%%%%%%%%%%%%%%%%%%%%%%%%%%%%%%%%%%%%%%%%%%%%%%%%%%%%%%%%%%%%%%%%%%%%%%%%%%%%%

%  Kopf und Fußzeilen -- links und rechts verschieden
\newcommand{\kopfbild}{\voffset7mm\includegraphics[width=25mm]{HTL3RLogo}}
\newcommand{\kopfHTL}{\sffamily{\textbf{\large{Projekthandbuch HTL3R}}}}

\usepackage[automark,footsepline,plainfootsepline]{scrlayer-scrpage}
\setkomafont{pageheadfoot}{\normalcolor\footnotesize\scshape}
\setkomafont{pagenumber}{\normalfont\normalsize}
\clearpairofpagestyles
\ihead{\headmark}
\ihead{\kopfbild}
\ohead{\kopfHTL}
\ifoot{\smaller{Höhere Technische Bundeslehranstalt Wien 3 | Rennweg 89b | 1030 Wien | \textcolor{orange}{www.htl.rennweg.at}}}
\ofoot{Seite \pagemark/\pageref{LastPage}}
\ModifyLayer[addvoffset=-.6ex]{scrheadings.foot.above.line}% Linie verschieben
\ModifyLayer[addvoffset=-.6ex]{plain.scrheadings.foot.above.line}% Linie verschieben
\setlength{\headheight}{32pt}

% alle Seiten mit Kopfzeile
%\renewcommand{\chapterpagestyle}{scrheadings}

%% Code Beispiele
%% eine Variante
\usepackage{listings}
\renewcommand{\lstlistingname}{\inputencoding{utf8}Listing}

\usepackage{tabularx}
\usepackage{scrhack}

\usepackage{array}
\newcommand\Tstrut{\rule{0pt}{3.2ex}}         % = `top' strut
\newcommand\Bstrut{\rule[-1.5ex]{0pt}{0pt}}   % = `bottom' strut

\newenvironment{nstabbing}
	{\setlength{\topsep}{-\parskip}
		\setlength{\partopsep}{-\parskip}
		\tabbing}
	{\endtabbing}

\usepackage{titlesec}
% \titleformat{?Überschriftenklasse?}[Absatzformatierung?]{?Textformatierung?} {?Nummerierung?}{?Abstand zwischen Nummerierung und Überschriftentext?}{?Code vor der Überschrift?}[?Code nach der Überschrift?]
\titleformat{\section}[hang]{\Large\bfseries\sffamily}{\thesection\quad}{-1.2ex}{}
\titleformat{\subsection}[hang]{\large\bfseries\sffamily}{\thesubsection\quad}{-1.2ex}{}
\titleformat{\subsubsection}[hang]{\large\bfseries\sffamily}{\thesubsubsection\quad}{-1.2ex}{}
\titleformat{\paragraph}[hang]{\large\bfseries\sffamily}{\theparagraph\quad}{-1.2ex}{}

% \titlespacing{?Überschriftenklasse?}{?Linker Einzug?}{?Platz oberhalb?}{?Platz unterhalb?}[?rechter Einzug?]
\titlespacing{\section}{0pt}{6pt}{6pt}
\titlespacing{\subsection}{0pt}{6pt}{0pt}
\titlespacing{\subsubsection}{0pt}{6pt}{0pt}
\titlespacing{\paragraph}{0pt}{6pt}{0pt}

%% sollte das letzte Package sein
\usepackage[unicode=true,
bookmarks=true,bookmarksnumbered=false,bookmarksopen=false,
breaklinks=true,pdfborder={0 0 0},backref=false,colorlinks=false]
{hyperref}
\hypersetup{pdftitle={Beispiel ITP Zieldefinition Wasserfall},
	pdfauthor={Wer auch immer},
	pdfsubject={ITP},
	pdfkeywords={4CN, ITP}}
\urlstyle{same} % don't use monospace font for urls

% Auch Fußnoten bündig ausrichten
\deffootnote[]{1em}{1em}{\textsuperscript{\thefootnotemark\ }}
%% setup
\sloppy % weniger Meldungen
\voffset7mm % etwas nach unten

%%%%%%%%%%%%%%%%%%%%%%%%%%%%%%%%%%%%%%%%%%%%%%%%%%%%%%%%%%%%%%%%%%%%%%%%%%%%%%%%%%
\begin{document}
%% schöner: 10000 -- gar keine, 1000 als Mittelweg
\clubpenalty = 1000 % Schusterjungen verhindern
\widowpenalty = 1000 % Hurenkinder verhindern
\displaywidowpenalty = 1000

{\sffamily{\textbf{\LARGE{\textcolor{orange}{Projektziele}}}}}\\
\noindent\rule{\textwidth}{0.1pt}
\begin{nstabbing}
	\hspace{4cm}\=\hspace{4cm}\=\hspace{4cm}\=\kill
	Projekttitel: \> \textbf{Projekttitel eingeben}\\
	Auftraggeber*in: \> \textbf{Vorname Nachname / Unternehmen (eine Person)}\\
	Auftragnehmer*in: \> \textbf{Vorname Nachname Projektleiter*in / Unternehmen}\\
	Schuljahr: \> \textbf{202X/2X}
	\> Klasse: \> \textbf{XXX}\\
\end{nstabbing}
{\smaller
	\begin{tabularx}{\textwidth}{l l l l}
	\hline
	\textbf{Version} & \textbf{Datum} & \textbf{Autorin/Autor} & \textbf{Änderung}\Tstrut  \\
	v1.0 & 03.03.2024 & Vorname Nachname & Beschreibung der Änderung, z.B. Erstellung...\Bstrut \\
	\hline
	\end{tabularx}
}

\section{Projektausgangssituation}
Kurze Beschreibung der Voraussetzungen und Rahmenbedingungen des Projekts. Wie ist es zu dem Projekt gekommen, welche Rand- und Rahmenbedingungen gibt es?

\section{Projektidee}
Epische Beschreibung der Projektidee. Es muss Außenstehenden die Idee klar skizziert werden. 
Unter Umständen können Grafiken hier helfen. Die Projektidee ist ein sehr wichtiges Dokument, weil dieser Text während und nach Abschluss eines Projektes sehr oft gebraucht und verwendet wird.

\section{Projektziele}
\subsection{Ziele}
SMARTE Beschreibung der Ziele des Projekts als Zustand nach Abschluss des Projekts. Die Zielformulierung beschreibt das gesamte Projekt vollständig, d.h. wenn man die Zielformulierung liest, benötigt man keine zusätzlichen Infos mehr. 
\begin{itemize}
	\item[MZ01] Zielbeschreibung Muss-Ziel 01
	\item[MZ02] Bsp. Event: Es steht ein kostenloses Buffet für 100 Gäste mit 3 unterschiedlichen warmen Hauptspeisen im Vorraum des Konferenzsaals in der Zeit von 13 bis 15 Uhr am Eventtag zur Verfügung. 
	\item[MZ03] Bsp. Medientechnik: Das User Interface hat die CD-Richtlinien des Unternehmens xy (siehe cd\_dokument.pdf) vollständig berücksichtigt ODER Der Webshop akzeptiert Zahlungen per Kreditkarte (Visa, Mastercard). 
	\item[MZ04] Bsp. Netzwerktechnik: Für die Website wird ein Audit gemäß der „OWASP Top Ten“ durchgeführt ODER Die Website verfügt über ein SSL-Zertifikat. 
	\item[MZ05] Bsp. Mechatronik: Der Farbton des Lichts wird mit einem Sensor gemessen und über ein Display ausgegeben ODER Das Akku des Gerätes kann über ein induktives Ladepad geladen werden. 
\end{itemize}

\subsection{Optionale Ziele}
\begin{itemize}
	\item[OZ01] Zielbeschreibung Optionales-Ziel 01
	\item[OZ02] Zielbeschreibung Optionales-Ziel 02
	\item[OZ03] Zielbeschreibung Optionales-Ziel 03
\end{itemize}

\subsection{Nicht Ziele}
NICHT ZIELE werden positiv formuliert und sollen das Projekt eingrenzen. Sie dienen vor allem zur Sicherheit des Auftragnehmers, weil explizit formuliert wird, was NICHT geliefert wird. Es werden Features ausgeschlossen, die sich ein Auftraggeber im Zuge des Projekts erwarten könnte.
\begin{itemize}
	\item[NZ01] Zielbeschreibung Nicht-Ziel 01
	\item[NZ02] Bsp. Event: Geschirr, Besteck und Tischtücher sind vom Projektteam gekauft worden.
\end{itemize}

\end{document}