\documentclass[
    headings=optiontotocandhead,% Erweiterung für das optionale Argument der
                                % Gliederungsbefehle aktiviert.
    twoside,
    numbers=noenddot,% Keine Punkte am Ende der Gliederungsnummern und davon
                     % abgeleiteten Nummern
    toc=flat, %Flache TOC --- kann man anpassen (auskommentieren)
    12pt, % Schriftgröße
    titlepage, % es wird eine Titelseite verwendet
    parskip=full, % Abstand zwischen Absätzen (ganze Zeile)
    listof=totoc, % Verzeichnisse im Inhaltsverzeichnis aufführen
    listof=flat, % mehr Abstand für grosse Zahlen
    numbers=noenddot, % kein Punkt am Ende bei Nummern
    %%enlargefirstpage,% Gibt es bei scrartcl nicht!!!!
    bibliography=totoc, % Literaturverzeichnis im Inhaltsverzeichnis aufführen
    %index=totoc, % Index im Inhaltsverzeichnis aufführen
    %captions=tableheading, % Beschriftung von Tabellen für Ausgabe oberhalb
                           % der Tabelle formatieren
    %draft % Status des Dokuments (final/draft) draft hinzufügen zum anziegen
    %%der zeilen ende
    a4paper,DIV=14,
    BCOR=15mm,
    % captions=tablesignature,
]{scrbook}

\setcounter{secnumdepth}{3}

\usepackage[T1]{fontenc}
\usepackage[utf8]{inputenc}

\usepackage[english, ngerman]{babel, varioref} % your native language must be the last one!!

\usepackage{lastpage}
\usepackage{listings}
\usepackage{blindtext}

%% Aufzählungen nicht so weit einrücken
\usepackage[inline]{enumitem}
%\setitemize{leftmargin=*}
% Listen etwas wenige einrücken, erfordert enumitem
\setitemize{leftmargin=*}

\usepackage{lmodern}

\usepackage{xspace}

\usepackage{graphicx}

%%? \usepackage{textcomp}
\usepackage[hyphens]{url}
\usepackage{makeidx}
\makeindex
%%? \usepackage{graphicx}
\usepackage[numbers]{natbib}
\PassOptionsToPackage{normalem}{ulem}
\usepackage{ulem}

\usepackage{needspace}

\setlength\partopsep{0.5ex}%schoenere Listen
\usepackage[bottom]{footmisc}%fussnote ganz unten

\usepackage[]{microtype}
\UseMicrotypeSet[protrusion]{basicmath} % disable protrusion for tt fonts

\usepackage{multirow}   % Allows table elements to span several rows.
\usepackage{booktabs}   % Improves the typesettings of tables.
\usepackage{subcaption} % Allows the use of subfigures and enables their referencing.
\usepackage[ruled,linesnumbered,algochapter]{algorithm2e} % Enables the writing of pseudo code.
\usepackage[usenames,dvipsnames,table]{xcolor} % Allows the definition and use of colors. This package has to be included before tikz.
\usepackage{nag}       % Issues warnings when best practices in writing LaTeX documents are violated.
\usepackage{todonotes} % Provides tooltip-like todo notes.

\usepackage{color}
\usepackage[binary-units]{siunitx}

%% Override default figure placement To be within the flow of the text rather
%% than on it's own page.
% \usepackage{float}
% \makeatletter
% \def\fps@figure{H}
% \makeatother

%% bei vielen Bildern o.ä sinnvoll: Seite muss nicht bis ganz unten gefüllt werden
% \raggedbottom

%\usepackage{footbib} %  footcite, needs other tooling
%% for pandoc2 images
\makeatletter
\def\maxwidth{\ifdim\Gin@nat@width>\linewidth\linewidth\else\Gin@nat@width\fi}
\def\maxheight{\ifdim\Gin@nat@height>\textheight\textheight\else\Gin@nat@height\fi}
\makeatother
% Scale images if necessary, so that they will not overflow the page
% margins by default, and it is still possible to overwrite the defaults
% using explicit options in \includegraphics[width, height, ...]{}
\setkeys{Gin}{width=\maxwidth,height=\maxheight,keepaspectratio}

%% bessere Suche im PDF
\input{glyphtounicode}
\pdfgentounicode=1
%%%%%%%%%%%%%%%%%%%%%%%%%%%%%%%%%%%%%%%%%%%%%%%%%%%%%%%%%%%%%%%%%%%%%%%%%%%%%%%%%%

%  Kopf und Fußzeilen -- links und rechts verschieden
\newcommand{\kopfseitenummer}{{\bfseries \thepage}}
\newcommand{\kopfkapl}{{\bfseries\leftmark}}
\newcommand{\kopfkapr}{{\bfseries\rightmark}}
\newcommand{\kopfbild}{\voffset7mm\includegraphics[width=25mm]{HTL3RLogoRGB}}
\newcommand{\kopfHTL}{Höhere Technische Bundeslehranstalt Wien 3, \\Rennweg 	Abteilung für Informationstechnologie}

\usepackage[automark,headsepline,footsepline,plainfootsepline]{scrlayer-scrpage}
%\automark[chapter]{chapter}% Eventuell wenn doppelseitig
\setkomafont{pageheadfoot}{\normalcolor\footnotesize\scshape}
\setkomafont{pagenumber}{\normalfont\normalsize}
\clearpairofpagestyles
\ihead{\headmark}
\ohead{\kopfbild}
\ifoot{\kapitelautor}
\ofoot{\pagemark}
\ModifyLayer[addvoffset=-.6ex]{scrheadings.foot.above.line}% Linie verschieben
\ModifyLayer[addvoffset=-.6ex]{plain.scrheadings.foot.above.line}% Linie verschieben
\setlength{\headheight}{32pt}

% alle Seiten mit Kopfzeile
\renewcommand{\chapterpagestyle}{scrheadings}

%% Kapitel - aufwändige Kapitelüberschriften
%Options: Sonny, Lenny, Glenn, Conny, Rejne, Bjarne, Bjornstrup
%\usepackage[Bjornstrup]{fncychap}
% Alternative:
%\usepackage{titlesec}

% Verzeichnisse - aufwändiger
%\usepackage{tocloft}


%% Code Beispiele
%% eine Variante
\usepackage{listings}
\renewcommand{\lstlistingname}{\inputencoding{utf8}Listing}
%% andere Variante
%\usepackage{minted}
%\setminted{
%  linenos,
%  frame=lines,
%  framesep=2mm,
%  breaklines=true
%}
% Beispiel
%\begin{listing}[H]
%\begin{minted}{bash}
%...
%\end{minted}
%\caption{Beschreibung}
%\end{listing}
%% dritte Variante
% mit/für pandoc
\input{text/00_pandoclisting.tex}

%% should be last packages
\usepackage{scrhack}

%% glossar
% kann man löschen falls kein Glossar gebraucht
\usepackage[acronym, toc]{glossaries}
\makeglossaries
% ac
%% https://de.overleaf.com/learn/latex/Glossaries

%% Makros zur schnelle Definition von Acronymen und Glossareinrägen

%% Copyright Lorenz Stechauner, 2019

%%%%%%%%%%%%%%%%

    % 1 -> key
    % 2 -> name <---- which is also the short
    % 3 -> pluralname
    % 4 -> long
    % 5 -> longplural
    % 6 -> description

\newcommand*{\newacr}[5]{
    \newglossaryentry{#1}{
        type=\acronymtype,
        name={#2},
        short={#2},
        shortplural={#3},
        plural={#3},
        long={#4},
        longplural={#5},
        description={#4},
        first={#4 (#2)},
        firstplural={#5 (#3)}
   }
}
\newcommand*{\ac}[1]{\gls{#1}}
\newcommand*{\acs}[1]{\acrshort{#1}}
\newcommand*{\acl}[1]{\acrlong{#1}}
\newcommand*{\acf}[1]{\acrfull{#1}}
\newcommand*{\acp}[1]{\glsplural{#1}}
\newcommand*{\acsp}[1]{\acrshortpl{#1}}
\newcommand*{\aclp}[1]{\acrlongpl{#1}}
\newcommand*{\acfp}[1]{\acrfullpl{#1}}
\newcommand*{\Ac}[1]{\Gls{#1}}
\newcommand*{\Acs}[1]{\Acrshort{#1}}
\newcommand*{\Acl}[1]{\Acrlong{#1}}
\newcommand*{\Acf}[1]{\Acrfull{#1}}
\newcommand*{\Acp}[1]{\Glsplural{#1}}
\newcommand*{\Acsp}[1]{\Acrshortpl{#1}}
\newcommand*{\Aclp}[1]{\Acrlongpl{#1}}
\newcommand*{\Acfp}[1]{Acrfullpl{#1}}

\newcommand*{\newacrgls}[6]{
    \longnewglossaryentry{{#1}_gls}{name={#4}, see=[Abkürzungsverzeichnis:]{#1}}{#6}
    \newglossaryentry{#1}{
        type=\acronymtype,
        name={#2},
        short={#2},
        plural={#3},
        long={#4},
        longplural={#5},
        description={#4},
        first={#4 (#2)},
        firstplural={#5 (#3)},
        see=[Glossar:]{{#1}_gls}
    }
}
\newcommand*{\glac}[1]{\gls{#1}\glsadd{{#1}_gls}}
\newcommand*{\glacs}[1]{\acrshort{#1}\glsadd{{#1}_gls}}
\newcommand*{\glacl}[1]{\acrlong{#1}\glsadd{{#1}_gls}}
\newcommand*{\glacf}[1]{\acrfull{#1}\glsadd{{#1}_gls}}
\newcommand*{\glacp}[1]{\glsplural{#1}\glsadd{{#1}_gls}}
\newcommand*{\glacsp}[1]{\acrshortpl{#1}\glsadd{{#1}_gls}}
\newcommand*{\glaclp}[1]{\acrlongpl{#1}\glsadd{{#1}_gls}}
\newcommand*{\Glac}[1]{\Gls{#1}\glsadd{{#1}_gls}}
\newcommand*{\Glacs}[1]{\Acrshort{#1}\glsadd{{#1}_gls}}
\newcommand*{\Glacl}[1]{\Acrlong{#1}\glsadd{{#1}_gls}}
\newcommand*{\Glacf}[1]{\Acrfull{#1}\glsadd{#1}_gls}
\newcommand*{\Glacp}[1]{\Glsplural{#1}\glsadd{{#1}_gls}}
\newcommand*{\Glacsp}[1]{\Acrshortpl{#1}\glsadd{{#1}_gls}}
\newcommand*{\Glaclp}[1]{\Acrlongpl{#1}\glsadd{{#1}_gls}}

\newcommand*{\newgls}[4]{
    \longnewglossaryentry{#1}{name={#2}, plural={#3}}{#4}
}
\newcommand*{\gl}[1]{\gls{#1}}
\newcommand*{\glp}[1]{\glsplural{#1}}

%%%%%%%%%%%%%%%%


\newcommand*{\incabbpng}[4]{
    % Label, Beschreibung, Pfad, Quelle, Genaue Quelle
    \begin{figure}
    \centering
    \includegraphics[width=0.75\textwidth,bb={#4}]{#3}
    \caption{#2}
    \label{#1}
    \end{figure}
}


\newcommand*{\incabbpngq}[6]{
    % Label, Beschreibung, Pfad, Quelle, Genaue Quelle
    \begin{figure}
    \centering
    \includegraphics[width=0.75\textwidth,bb={#6}]{#3}
    \caption{#2 \cite{#4}}
    \label{#1}
    \end{figure}
}

\newcommand*{\incabbq}[5]{
    % Label, Beschreibung, Pfad, Quelle, Genaue Quelle
    \begin{figure}
    \centering
    \includegraphics[width=0.75\textwidth]{#3}
    \caption{#2 \cite{#4}}
    \label{#1}
    \end{figure}
}

\newcommand*{\incabbsvgq}[5]{
    % Label, Beschreibung, Pfad, Quelle, Genaue Quelle
    \begin{figure}
    \centering
    \includesvg[width=0.75\textwidth]{#3}
    \caption{#2 \cite{#4}}
    \label{#1}
    \end{figure}
}

\newcommand*{\incabb}[3]{
    % Label, Beschreibung, Pfad
    \begin{figure}
    \centering
    \includegraphics[width=0.75\textwidth]{#3}
    \caption{#2}
    \label{#1}
    \end{figure}
}

\newcommand*{\incabbh}[3]{
    % Label, Beschreibung, Pfad
    \begin{figure}[H]
    \centering
    \includegraphics[width=0.75\textwidth]{#3}
    \caption{#2}
    \label{#1}
    \end{figure}
}

\newcommand*{\incabbsvg}[3]{
    % Label, Beschreibung, Pfad
    \begin{figure}
    \centering
    \includesvg[width=0.75\textwidth]{#3}
    \caption{#2}
    \label{#1}
    \end{figure}
}


%%%%%%%%%%%%%%%%

%% in diplomarbeit.tex:
%%\usepackage[acronym, toc]{glossaries}
%%\makeglossaries
%%% ac
%% https://de.overleaf.com/learn/latex/Glossaries

%% Makros zur schnelle Definition von Acronymen und Glossareinrägen

%% Copyright Lorenz Stechauner, 2019

%%%%%%%%%%%%%%%%

    % 1 -> key
    % 2 -> name <---- which is also the short
    % 3 -> pluralname
    % 4 -> long
    % 5 -> longplural
    % 6 -> description

\newcommand*{\newacr}[5]{
    \newglossaryentry{#1}{
        type=\acronymtype,
        name={#2},
        short={#2},
        shortplural={#3},
        plural={#3},
        long={#4},
        longplural={#5},
        description={#4},
        first={#4 (#2)},
        firstplural={#5 (#3)}
   }
}
\newcommand*{\ac}[1]{\gls{#1}}
\newcommand*{\acs}[1]{\acrshort{#1}}
\newcommand*{\acl}[1]{\acrlong{#1}}
\newcommand*{\acf}[1]{\acrfull{#1}}
\newcommand*{\acp}[1]{\glsplural{#1}}
\newcommand*{\acsp}[1]{\acrshortpl{#1}}
\newcommand*{\aclp}[1]{\acrlongpl{#1}}
\newcommand*{\acfp}[1]{\acrfullpl{#1}}
\newcommand*{\Ac}[1]{\Gls{#1}}
\newcommand*{\Acs}[1]{\Acrshort{#1}}
\newcommand*{\Acl}[1]{\Acrlong{#1}}
\newcommand*{\Acf}[1]{\Acrfull{#1}}
\newcommand*{\Acp}[1]{\Glsplural{#1}}
\newcommand*{\Acsp}[1]{\Acrshortpl{#1}}
\newcommand*{\Aclp}[1]{\Acrlongpl{#1}}
\newcommand*{\Acfp}[1]{Acrfullpl{#1}}

\newcommand*{\newacrgls}[6]{
    \longnewglossaryentry{{#1}_gls}{name={#4}, see=[Abkürzungsverzeichnis:]{#1}}{#6}
    \newglossaryentry{#1}{
        type=\acronymtype,
        name={#2},
        short={#2},
        plural={#3},
        long={#4},
        longplural={#5},
        description={#4},
        first={#4 (#2)},
        firstplural={#5 (#3)},
        see=[Glossar:]{{#1}_gls}
    }
}
\newcommand*{\glac}[1]{\gls{#1}\glsadd{{#1}_gls}}
\newcommand*{\glacs}[1]{\acrshort{#1}\glsadd{{#1}_gls}}
\newcommand*{\glacl}[1]{\acrlong{#1}\glsadd{{#1}_gls}}
\newcommand*{\glacf}[1]{\acrfull{#1}\glsadd{{#1}_gls}}
\newcommand*{\glacp}[1]{\glsplural{#1}\glsadd{{#1}_gls}}
\newcommand*{\glacsp}[1]{\acrshortpl{#1}\glsadd{{#1}_gls}}
\newcommand*{\glaclp}[1]{\acrlongpl{#1}\glsadd{{#1}_gls}}
\newcommand*{\Glac}[1]{\Gls{#1}\glsadd{{#1}_gls}}
\newcommand*{\Glacs}[1]{\Acrshort{#1}\glsadd{{#1}_gls}}
\newcommand*{\Glacl}[1]{\Acrlong{#1}\glsadd{{#1}_gls}}
\newcommand*{\Glacf}[1]{\Acrfull{#1}\glsadd{#1}_gls}
\newcommand*{\Glacp}[1]{\Glsplural{#1}\glsadd{{#1}_gls}}
\newcommand*{\Glacsp}[1]{\Acrshortpl{#1}\glsadd{{#1}_gls}}
\newcommand*{\Glaclp}[1]{\Acrlongpl{#1}\glsadd{{#1}_gls}}

\newcommand*{\newgls}[4]{
    \longnewglossaryentry{#1}{name={#2}, plural={#3}}{#4}
}
\newcommand*{\gl}[1]{\gls{#1}}
\newcommand*{\glp}[1]{\glsplural{#1}}

%%%%%%%%%%%%%%%%


\newcommand*{\incabbpng}[4]{
    % Label, Beschreibung, Pfad, Quelle, Genaue Quelle
    \begin{figure}
    \centering
    \includegraphics[width=0.75\textwidth,bb={#4}]{#3}
    \caption{#2}
    \label{#1}
    \end{figure}
}


\newcommand*{\incabbpngq}[6]{
    % Label, Beschreibung, Pfad, Quelle, Genaue Quelle
    \begin{figure}
    \centering
    \includegraphics[width=0.75\textwidth,bb={#6}]{#3}
    \caption{#2 \cite{#4}}
    \label{#1}
    \end{figure}
}

\newcommand*{\incabbq}[5]{
    % Label, Beschreibung, Pfad, Quelle, Genaue Quelle
    \begin{figure}
    \centering
    \includegraphics[width=0.75\textwidth]{#3}
    \caption{#2 \cite{#4}}
    \label{#1}
    \end{figure}
}

\newcommand*{\incabbsvgq}[5]{
    % Label, Beschreibung, Pfad, Quelle, Genaue Quelle
    \begin{figure}
    \centering
    \includesvg[width=0.75\textwidth]{#3}
    \caption{#2 \cite{#4}}
    \label{#1}
    \end{figure}
}

\newcommand*{\incabb}[3]{
    % Label, Beschreibung, Pfad
    \begin{figure}
    \centering
    \includegraphics[width=0.75\textwidth]{#3}
    \caption{#2}
    \label{#1}
    \end{figure}
}

\newcommand*{\incabbh}[3]{
    % Label, Beschreibung, Pfad
    \begin{figure}[H]
    \centering
    \includegraphics[width=0.75\textwidth]{#3}
    \caption{#2}
    \label{#1}
    \end{figure}
}

\newcommand*{\incabbsvg}[3]{
    % Label, Beschreibung, Pfad
    \begin{figure}
    \centering
    \includesvg[width=0.75\textwidth]{#3}
    \caption{#2}
    \label{#1}
    \end{figure}
}


%%%%%%%%%%%%%%%%

%% in diplomarbeit.tex:
%%\usepackage[acronym, toc]{glossaries}
%%\makeglossaries
%%% ac
%% https://de.overleaf.com/learn/latex/Glossaries

%% Makros zur schnelle Definition von Acronymen und Glossareinrägen

%% Copyright Lorenz Stechauner, 2019

%%%%%%%%%%%%%%%%

    % 1 -> key
    % 2 -> name <---- which is also the short
    % 3 -> pluralname
    % 4 -> long
    % 5 -> longplural
    % 6 -> description

\newcommand*{\newacr}[5]{
    \newglossaryentry{#1}{
        type=\acronymtype,
        name={#2},
        short={#2},
        shortplural={#3},
        plural={#3},
        long={#4},
        longplural={#5},
        description={#4},
        first={#4 (#2)},
        firstplural={#5 (#3)}
   }
}
\newcommand*{\ac}[1]{\gls{#1}}
\newcommand*{\acs}[1]{\acrshort{#1}}
\newcommand*{\acl}[1]{\acrlong{#1}}
\newcommand*{\acf}[1]{\acrfull{#1}}
\newcommand*{\acp}[1]{\glsplural{#1}}
\newcommand*{\acsp}[1]{\acrshortpl{#1}}
\newcommand*{\aclp}[1]{\acrlongpl{#1}}
\newcommand*{\acfp}[1]{\acrfullpl{#1}}
\newcommand*{\Ac}[1]{\Gls{#1}}
\newcommand*{\Acs}[1]{\Acrshort{#1}}
\newcommand*{\Acl}[1]{\Acrlong{#1}}
\newcommand*{\Acf}[1]{\Acrfull{#1}}
\newcommand*{\Acp}[1]{\Glsplural{#1}}
\newcommand*{\Acsp}[1]{\Acrshortpl{#1}}
\newcommand*{\Aclp}[1]{\Acrlongpl{#1}}
\newcommand*{\Acfp}[1]{Acrfullpl{#1}}

\newcommand*{\newacrgls}[6]{
    \longnewglossaryentry{{#1}_gls}{name={#4}, see=[Abkürzungsverzeichnis:]{#1}}{#6}
    \newglossaryentry{#1}{
        type=\acronymtype,
        name={#2},
        short={#2},
        plural={#3},
        long={#4},
        longplural={#5},
        description={#4},
        first={#4 (#2)},
        firstplural={#5 (#3)},
        see=[Glossar:]{{#1}_gls}
    }
}
\newcommand*{\glac}[1]{\gls{#1}\glsadd{{#1}_gls}}
\newcommand*{\glacs}[1]{\acrshort{#1}\glsadd{{#1}_gls}}
\newcommand*{\glacl}[1]{\acrlong{#1}\glsadd{{#1}_gls}}
\newcommand*{\glacf}[1]{\acrfull{#1}\glsadd{{#1}_gls}}
\newcommand*{\glacp}[1]{\glsplural{#1}\glsadd{{#1}_gls}}
\newcommand*{\glacsp}[1]{\acrshortpl{#1}\glsadd{{#1}_gls}}
\newcommand*{\glaclp}[1]{\acrlongpl{#1}\glsadd{{#1}_gls}}
\newcommand*{\Glac}[1]{\Gls{#1}\glsadd{{#1}_gls}}
\newcommand*{\Glacs}[1]{\Acrshort{#1}\glsadd{{#1}_gls}}
\newcommand*{\Glacl}[1]{\Acrlong{#1}\glsadd{{#1}_gls}}
\newcommand*{\Glacf}[1]{\Acrfull{#1}\glsadd{#1}_gls}
\newcommand*{\Glacp}[1]{\Glsplural{#1}\glsadd{{#1}_gls}}
\newcommand*{\Glacsp}[1]{\Acrshortpl{#1}\glsadd{{#1}_gls}}
\newcommand*{\Glaclp}[1]{\Acrlongpl{#1}\glsadd{{#1}_gls}}

\newcommand*{\newgls}[4]{
    \longnewglossaryentry{#1}{name={#2}, plural={#3}}{#4}
}
\newcommand*{\gl}[1]{\gls{#1}}
\newcommand*{\glp}[1]{\glsplural{#1}}

%%%%%%%%%%%%%%%%


\newcommand*{\incabbpng}[4]{
    % Label, Beschreibung, Pfad, Quelle, Genaue Quelle
    \begin{figure}
    \centering
    \includegraphics[width=0.75\textwidth,bb={#4}]{#3}
    \caption{#2}
    \label{#1}
    \end{figure}
}


\newcommand*{\incabbpngq}[6]{
    % Label, Beschreibung, Pfad, Quelle, Genaue Quelle
    \begin{figure}
    \centering
    \includegraphics[width=0.75\textwidth,bb={#6}]{#3}
    \caption{#2 \cite{#4}}
    \label{#1}
    \end{figure}
}

\newcommand*{\incabbq}[5]{
    % Label, Beschreibung, Pfad, Quelle, Genaue Quelle
    \begin{figure}
    \centering
    \includegraphics[width=0.75\textwidth]{#3}
    \caption{#2 \cite{#4}}
    \label{#1}
    \end{figure}
}

\newcommand*{\incabbsvgq}[5]{
    % Label, Beschreibung, Pfad, Quelle, Genaue Quelle
    \begin{figure}
    \centering
    \includesvg[width=0.75\textwidth]{#3}
    \caption{#2 \cite{#4}}
    \label{#1}
    \end{figure}
}

\newcommand*{\incabb}[3]{
    % Label, Beschreibung, Pfad
    \begin{figure}
    \centering
    \includegraphics[width=0.75\textwidth]{#3}
    \caption{#2}
    \label{#1}
    \end{figure}
}

\newcommand*{\incabbh}[3]{
    % Label, Beschreibung, Pfad
    \begin{figure}[H]
    \centering
    \includegraphics[width=0.75\textwidth]{#3}
    \caption{#2}
    \label{#1}
    \end{figure}
}

\newcommand*{\incabbsvg}[3]{
    % Label, Beschreibung, Pfad
    \begin{figure}
    \centering
    \includesvg[width=0.75\textwidth]{#3}
    \caption{#2}
    \label{#1}
    \end{figure}
}


%%%%%%%%%%%%%%%%

%% in diplomarbeit.tex:
%%\usepackage[acronym, toc]{glossaries}
%%\makeglossaries
%%\input{text/00_Glossar.tex}




\newacr{ACL}{ACL}{ACLs}{Access Control List}{Access Control Lists}
\newacr{AD}{AD}{ADs}{Active Directory}{Active Directorys}
\newacr{API}{API}{APIs}{Application Programming Interface}{Application Programming Interfaces}
\newacr{ATA}{ATA}{ATAs}{Advanced Threat Analytics}{Advanced Threat Analytics}
\newacr{DoS}{DoS}{DoS'}{Denial of Service}{Denial of Services}
\newacr{DPI}{DPI}{DPIs}{Deep Packet Inspection}{Deep Packet Inspections}
\newacr{DB}{DB}{DBs}{Datenbank}{Datenbanken}
\newacr{DBMS}{DBMS}{DBMSs}{Datenbankmanagementsystem}{Datenbankmanagementsysteme}
\newacr{DNS}{DNS}{DNS}{Domain Name System}{Domain Name Systeme}
\newacr{DSGVO}{DSGVO}{DSGVOs}{Datenschutz-Grundverordnung}{Datenschutz-Grundverordnungen}
\newacr{DSG}{DSG}{DSGs}{Datenschutzgesetz}{Datenschutzgesetze}
\newacr{EDV}{EDV}{EDVs}{Elektronische Datenverarbeitung}{Elektronische Datenverarbeitungen}
\newacr{FQDN}{FQDN}{FQDNs}{Fully-Qualified Domain Name}{Fully-Qualified Domain Names}
\newacr{GPO}{GPO}{GPOs}{Group Policy Object}{Group Policy Objects}
\newacr{HTL}{HTL}{HTLs}{Höhere Technische Lehranstalt}{Höhere Technische Lehranstalten}
\newacr{HTTP}{HTTP}{HTTP}{Hypertext Transfer Protocol}{Hypertext Transfer Protocol}
\newacr{HTTPS}{HTTPS}{HTTPS}{Hypertext Transfer Protocol Secure}{Hypertext Transfer Protocol Secure}
\newacr{IP}{IP}{IPs}{Internet Protocol}{Internet Protocols}
\newacr{ISP}{ISP}{ISPs}{Internet Service Provider}{Internet Service Providers}
\newacr{ID}{ID}{IDs}{Identifikator}{Identifikatoren}
\newacr{IT}{IT}{ITs}{Informationstechnologie}{Informationstechnologien}
\newacr{JSON}{JSON}{JSONs}{JavaScript Object Notation}{JavaScript Object Notations}
\newacr{KI}{KI}{KIs}{Künstliche Intelligenz}{Künstliche Intelligenzen}
\newacr{KMU}{KMU}{KMUs}{kleines und mittleres Unternehmen}{kleine und mittlere Unternehmen}
\newacr{LAN}{LAN}{LANs}{Local Area Network}{Local Area Networks}
\newacr{MAC}{MAC}{MACs}{Media Access Control}{Media Access Controls}
\newacr{NAT}{NAT}{NATs}{Network Address Translation}{Network Address Translations}
\newacr{NGFW}{NGFW}{NGFWs}{Next Generation Firewall}{Next Generation Firewalls}
\newacr{NoSQL}{NoSQL}{NoSQLs}{Not only \acs{SQL}}{Not only \acsp{SQL}}
\newacr{NPS}{NPS}{NPSs}{Network Policy Server}{Network Policy Server}
\newacr{NTP}{NTP}{NTPs}{Network Time Protocol}{Network Time Protcols}
\newacr{OID}{OID}{OIDs}{Object Identifier}{Object Identifier}
\newacr{OSI}{OSI}{OSIs}{Open Systems Interconnection}{Open Systems Interconnections}
\newacr{OUI}{OUI}{OUIs}{Organizationally Unique Identifier}{Organizationally Unique Identifier}
\newacr{PDU}{PDU}{PDUs}{Protocol Data Unit}{Protocol Data Units}
\newacr{AQL}{AQL}{AQLs}{\textit{Argos-Query-Language}}{\textit{Argos-Query-Languages}}
\newacr{RFC}{RFC}{RFCs}{Request for Comments}{Request for Comments}
\newacr{SaaS}{SaaS}{SaaSs}{Software as a Service}{Software as a Services}
\newacr{SID}{SID}{SID}{Security Identifier}{Security Identifier}
\newacr{SQL}{SQL}{SQLs}{Structured Query Language}{Structured Query Languages}
\newacr{SME}{SME}{SMEs}{small and medium-sized enterprise}{small and medium-sized enterprises}
\newacr{SNMP}{SNMP}{SNMPs}{Simple Network Monitoring Protocol}{Simple Network Monitoring Protocols}
\newacr{SSL}{SSL}{SSLs}{Secure Socket Layer}{Secure Socket Layers}
\newacr{TCP}{TCP}{TCPs}{Transmission Control Protocol}{Transmission Control Protocols}
\newacr{TLS}{TLS}{TLSs}{Transport Layer Security}{Transport Layer Securities}
\newacr{TTL}{TTL}{TTLs}{Time to Live}{Times to Live}
\newacr{UDP}{UDP}{UDPs}{User Datagram Protocol}{User Datagram Protocols}
\newacr{VLAN}{VLAN}{VLANs}{Virtual \acs{LAN}}{Virtual \acsp{LAN}}
\newacr{VPN}{VPN}{VPNs}{Virtual Private Network}{Virtual Private Networks}
\newacr{WLAN}{WLAN}{WLANs}{Wireless \acs{LAN}}{Wireless \acsp{LAN}}
\newacr{WLC}{WLC}{WLCs}{\acs{WLAN}-Controller}{\acs{WLAN}-Controller}
\newacr{XML}{XML}{XMLs}{Extensible Markup Language}{Extensible Markup Languages}
\newacr{IPFIX}{IPFIX}{IPFIXs}{\acs{IP} Flow Information Export}{\acs{IP} Flow Information Exports}
\newacr{MIB}{MIB}{MIBs}{Management Information Base}{Management Information Bases}
\newacr{MO}{MO}{MOs}{Managed Object}{Managed Objects}
\newacr{ASN.1}{ASN.1}{ASN.1}{Abstract Syntax Notation One}{Abstract Syntax Notation One}
\newacr{ASCII}{ASCII}{ASCII}{American Standard Code for Information Interchange}{American Standard Code for Information Interchange}
\newacr{RAM}{RAM}{RAM}{Random Access Memory}{Random Access Memory}
\newacr{CPU}{CPU}{CPUs}{Central Processing Unit}{Central Processing Units}
\newacr{CLI}{CLI}{CLIs}{Command Line Interface}{Command Line Interfaces}
\newacr{LWAP}{LWAP}{LWAPs}{Light Weight \ac{AP}}{Light Weight \acp{AP}}
\newacr{VTY}{VTY}{VTYs}{Virtual Terminal Line}{Virtual Terminal Lines}
\newacr{SMTP}{SMTP}{SMTP}{Simple Mail Transfer Protocol}{Simple Mail Transfer Protocol}
\newacr{GUI}{GUI}{GUIs}{Graphical User Interface}{Graphical User Interfaces}
\newacr{IPS}{IPS}{IPSs}{Intrusion Prevention System}{Intrusion Prevention Systems}
\newacr{UI}{UI}{UIs}{User Interface}{User Interfaces}
\newacr{CSS}{CSS}{CSS}{Cascading Style Sheets}{Cascading Style Sheets}
\newacr{HTML}{HTML}{HTML}{Hypertext Markup Language}{Hypertext Markup Language}
\newacr{MIT}{MIT}{MIT}{Massachusetts Institute of Technology}{Massachusetts Institute of Technology}
\newacr{PHP}{PHP}{PHP}{PHP: Hypertext Preprocessor}{PHP: Hypertext Preprocessor}
\newacr{LR}{LR}{LR}{Links nach rechts, rechtsreduzierender Parser}{Links nach rechts, rechtsreduzierender Parser}
\newacr{CSV}{CSV}{CSV}{Comma-Separated Values}{Comma-Separated Values}
\newacr{NT}{NT}{NTs}{New Technology}{New Technologies}
\newacr{SIM}{SIM}{SIM}{Security Information Management}{Security Information Management}
\newacr{SEM}{SEM}{SEM}{Security Event Management}{Security Event Management}
\newacr{EU}{EU}{EU}{Europäische Union}{Europäische Union}
\newacr{DSB}{DSB}{DSB}{Datenschutzbehörde}{Datenschutzbehörden}
\newacr{AMP}{AMP}{AMPs}{Advanced Malware Protection}{Advanced Malware Protections}
\newacr{AP}{AP}{APs}{Access Point}{Access Points}
\newacr{ASA}{ASA}{ASAs}{Adaptive Security Appliance}{Adaptive Security Appliances}
\newacr{ITIL}{ITIL}{ITILs}{\acs{IT} Infrastructure Library}{\acs{IT} Infrastructure Libraries}
\newacr{COBIT}{COBIT}{COBIT}{Control Objectives for Information and Related Technologies}{Control Objectives for Information and Related Technologies}
\newacr{SOX}{SOX}{SOX}{Sarbanes-Oxley Act}{Sarbanes-Oxley Act}
\newacr{ISO}{ISO}{ISO}{Internationale Organisation für Normung}{Internationale Organisation für Normung}
\newacr{IDS}{IDS}{IDSs}{Intrusion Detection System}{Intrusion Detection Systems}
\newacr{AWS}{AWS}{AWS}{Amazon Web Services}{Amazon Web Services}
\newacr{MacOS}{MacOS}{MacOS}{Macintosh Operating System}{Macintosh Operating System}
\newacr{CIM}{CIM}{CIMs}{Common Information Model}{Common Information Models}
\newacr{SPL}{SPL}{SPLs}{Search Processing Language}{Search Processing Languages}
\newacr{VM}{VM}{VMs}{Virtual Machine}{Virtual Machines}
\newacr{CMDB}{CMDB}{CMDBs}{ Configuration Management Database}{ Configuration Management Databases}
\newacr{PC}{PC}{PCs}{Personal Computer}{Personal Computers}


% Als Beschreibung für den Glossareintrag ist der erste Satzt in Wikipedia immer ziemlich hilfreich - glac

\newacrgls{SSID}{SSID}{SSIDs}{Service Set Identifier}{Service Set Identifiers}{ist ein frei wählbarer Name eines Service Sets, durch den es ansprechbar wird. Da diese Kennung oftmals manuell von einem Benutzer in Geräte eingegeben werden muss, ist sie oft eine Zeichenkette, die für Menschen leicht lesbar ist, und sie wird daher allgemein als (Funk-)Netzwerkname des \acsp{WLAN} bezeichnet.}
\newacrgls{SIEM}{SIEM}{SIEMs}{Security Information and Event Management}{Security Information and Event Managements}{kombiniert die zwei Konzepte \acf{SIM} und \acf{SEM} für die Echtzeitanalyse von Sicherheitsalarmen aus den Quellen Anwendungen und Netzwerkkomponenten.}
\newacrgls{LALR}{LALR}{LALR}{Look-ahead \acs{LR} Parser}{Look-ahead \acs{LR} Parser}{Look-ahead \acs{LR} Parser, wobei die Zahl in der Klammer die Anzahl der vorausschauenden Felder beschreibt.}
\newacrgls{RMSProp}{RMSProp}{RMSProp}{Root Mean Square Propagation}{Root Mean Square Propagation}{Ein Optimizer, der die Lernrate für jedes Gewicht dynamisch adaptiert.}
\newacrgls{Adam}{Adam}{Adam}{Adaptive Moment Estimation}{Adaptive Moment Estimation}{Eine neuere Version des \glacs{RMSProp} Optimizers, der performanter ist und auch jedes Gewicht dynamisch adaptiert.}
\newacrgls{LSTM}{LSTM}{LSTM}{Long short-term memory}{Long short-term memories}{Long short-term memory (langes Kurzzeitgedächtnis) ist eine Technik, die dafür sorgt, dass Neurale Netzwerke ein Gedächtnis haben.}

% gl

\newgls{MongoDB}{MongoDB}{MongoDBs}{eine dokumentenorientierte \acs{NoSQL}-\acl{DB}.\newline\href{https://www.mongodb.com/}{https://www.mongodb.com/}}
\newgls{Python}{Python}{Pythons}{eine universelle, üblicherweise interpretierte höhere Programmiersprache.\newline\href{https://www.python.org/}{https://www.python.org/}}
\newgls{Index}{Index}{Indizes}{eine von der Datenstruktur getrennte Indexstruktur in einer Datenbank, die die Suche und das Sortieren nach bestimmten Feldern beschleunigt.}
\newgls{OSI-Modell}{OSI\glsadd{OSI}-Modell}{OSI\glsadd{OSI}-Modelle}{ein Referenzmodell für Netzwerkprotokolle als Schichtenarchitektur.}
\newgls{OSI-Schicht}{OSI\glsadd{OSI}-Schicht}{OSI\glsadd{OSI}-Schichten}{Das \gl{OSI-Modell} hat sieben Schichten: 1 -- Bitübertragung (Physical), 2 -- Sicherung (Data Link), 3 -- Vermittlung-/Paket (Network), 4 -- Transport (Transport), 5 -- Sitzung (Session), 6 -- Darstellung (Presentation), 7 -- Anwendung (Application)}
\newgls{Daemon}{Daemon}{Daemons}{bezeichnet unter Unix oder unixartigen Systemen ein Programm, das im Hintergrund abläuft und bestimmte Dienste zur Verfügung stellt.}
\newgls{Thread}{Thread}{Threads}{bezeichnet einen Ausführungsstrang oder eine Ausführungsreihenfolge in der Abarbeitung eines Programms. Ein Thread ist Teil eines Prozesses.}
\newgls{Socket}{Socket}{Sockets}{ist ein vom Betriebssystem bereitgestelltes Objekt, das als Kommunikationsendpunkt dient. Ein Programm verwendet Sockets, um Daten mit anderen Programmen auszutauschen.}





\newacr{ACL}{ACL}{ACLs}{Access Control List}{Access Control Lists}
\newacr{AD}{AD}{ADs}{Active Directory}{Active Directorys}
\newacr{API}{API}{APIs}{Application Programming Interface}{Application Programming Interfaces}
\newacr{ATA}{ATA}{ATAs}{Advanced Threat Analytics}{Advanced Threat Analytics}
\newacr{DoS}{DoS}{DoS'}{Denial of Service}{Denial of Services}
\newacr{DPI}{DPI}{DPIs}{Deep Packet Inspection}{Deep Packet Inspections}
\newacr{DB}{DB}{DBs}{Datenbank}{Datenbanken}
\newacr{DBMS}{DBMS}{DBMSs}{Datenbankmanagementsystem}{Datenbankmanagementsysteme}
\newacr{DNS}{DNS}{DNS}{Domain Name System}{Domain Name Systeme}
\newacr{DSGVO}{DSGVO}{DSGVOs}{Datenschutz-Grundverordnung}{Datenschutz-Grundverordnungen}
\newacr{DSG}{DSG}{DSGs}{Datenschutzgesetz}{Datenschutzgesetze}
\newacr{EDV}{EDV}{EDVs}{Elektronische Datenverarbeitung}{Elektronische Datenverarbeitungen}
\newacr{FQDN}{FQDN}{FQDNs}{Fully-Qualified Domain Name}{Fully-Qualified Domain Names}
\newacr{GPO}{GPO}{GPOs}{Group Policy Object}{Group Policy Objects}
\newacr{HTL}{HTL}{HTLs}{Höhere Technische Lehranstalt}{Höhere Technische Lehranstalten}
\newacr{HTTP}{HTTP}{HTTP}{Hypertext Transfer Protocol}{Hypertext Transfer Protocol}
\newacr{HTTPS}{HTTPS}{HTTPS}{Hypertext Transfer Protocol Secure}{Hypertext Transfer Protocol Secure}
\newacr{IP}{IP}{IPs}{Internet Protocol}{Internet Protocols}
\newacr{ISP}{ISP}{ISPs}{Internet Service Provider}{Internet Service Providers}
\newacr{ID}{ID}{IDs}{Identifikator}{Identifikatoren}
\newacr{IT}{IT}{ITs}{Informationstechnologie}{Informationstechnologien}
\newacr{JSON}{JSON}{JSONs}{JavaScript Object Notation}{JavaScript Object Notations}
\newacr{KI}{KI}{KIs}{Künstliche Intelligenz}{Künstliche Intelligenzen}
\newacr{KMU}{KMU}{KMUs}{kleines und mittleres Unternehmen}{kleine und mittlere Unternehmen}
\newacr{LAN}{LAN}{LANs}{Local Area Network}{Local Area Networks}
\newacr{MAC}{MAC}{MACs}{Media Access Control}{Media Access Controls}
\newacr{NAT}{NAT}{NATs}{Network Address Translation}{Network Address Translations}
\newacr{NGFW}{NGFW}{NGFWs}{Next Generation Firewall}{Next Generation Firewalls}
\newacr{NoSQL}{NoSQL}{NoSQLs}{Not only \acs{SQL}}{Not only \acsp{SQL}}
\newacr{NPS}{NPS}{NPSs}{Network Policy Server}{Network Policy Server}
\newacr{NTP}{NTP}{NTPs}{Network Time Protocol}{Network Time Protcols}
\newacr{OID}{OID}{OIDs}{Object Identifier}{Object Identifier}
\newacr{OSI}{OSI}{OSIs}{Open Systems Interconnection}{Open Systems Interconnections}
\newacr{OUI}{OUI}{OUIs}{Organizationally Unique Identifier}{Organizationally Unique Identifier}
\newacr{PDU}{PDU}{PDUs}{Protocol Data Unit}{Protocol Data Units}
\newacr{AQL}{AQL}{AQLs}{\textit{Argos-Query-Language}}{\textit{Argos-Query-Languages}}
\newacr{RFC}{RFC}{RFCs}{Request for Comments}{Request for Comments}
\newacr{SaaS}{SaaS}{SaaSs}{Software as a Service}{Software as a Services}
\newacr{SID}{SID}{SID}{Security Identifier}{Security Identifier}
\newacr{SQL}{SQL}{SQLs}{Structured Query Language}{Structured Query Languages}
\newacr{SME}{SME}{SMEs}{small and medium-sized enterprise}{small and medium-sized enterprises}
\newacr{SNMP}{SNMP}{SNMPs}{Simple Network Monitoring Protocol}{Simple Network Monitoring Protocols}
\newacr{SSL}{SSL}{SSLs}{Secure Socket Layer}{Secure Socket Layers}
\newacr{TCP}{TCP}{TCPs}{Transmission Control Protocol}{Transmission Control Protocols}
\newacr{TLS}{TLS}{TLSs}{Transport Layer Security}{Transport Layer Securities}
\newacr{TTL}{TTL}{TTLs}{Time to Live}{Times to Live}
\newacr{UDP}{UDP}{UDPs}{User Datagram Protocol}{User Datagram Protocols}
\newacr{VLAN}{VLAN}{VLANs}{Virtual \acs{LAN}}{Virtual \acsp{LAN}}
\newacr{VPN}{VPN}{VPNs}{Virtual Private Network}{Virtual Private Networks}
\newacr{WLAN}{WLAN}{WLANs}{Wireless \acs{LAN}}{Wireless \acsp{LAN}}
\newacr{WLC}{WLC}{WLCs}{\acs{WLAN}-Controller}{\acs{WLAN}-Controller}
\newacr{XML}{XML}{XMLs}{Extensible Markup Language}{Extensible Markup Languages}
\newacr{IPFIX}{IPFIX}{IPFIXs}{\acs{IP} Flow Information Export}{\acs{IP} Flow Information Exports}
\newacr{MIB}{MIB}{MIBs}{Management Information Base}{Management Information Bases}
\newacr{MO}{MO}{MOs}{Managed Object}{Managed Objects}
\newacr{ASN.1}{ASN.1}{ASN.1}{Abstract Syntax Notation One}{Abstract Syntax Notation One}
\newacr{ASCII}{ASCII}{ASCII}{American Standard Code for Information Interchange}{American Standard Code for Information Interchange}
\newacr{RAM}{RAM}{RAM}{Random Access Memory}{Random Access Memory}
\newacr{CPU}{CPU}{CPUs}{Central Processing Unit}{Central Processing Units}
\newacr{CLI}{CLI}{CLIs}{Command Line Interface}{Command Line Interfaces}
\newacr{LWAP}{LWAP}{LWAPs}{Light Weight \ac{AP}}{Light Weight \acp{AP}}
\newacr{VTY}{VTY}{VTYs}{Virtual Terminal Line}{Virtual Terminal Lines}
\newacr{SMTP}{SMTP}{SMTP}{Simple Mail Transfer Protocol}{Simple Mail Transfer Protocol}
\newacr{GUI}{GUI}{GUIs}{Graphical User Interface}{Graphical User Interfaces}
\newacr{IPS}{IPS}{IPSs}{Intrusion Prevention System}{Intrusion Prevention Systems}
\newacr{UI}{UI}{UIs}{User Interface}{User Interfaces}
\newacr{CSS}{CSS}{CSS}{Cascading Style Sheets}{Cascading Style Sheets}
\newacr{HTML}{HTML}{HTML}{Hypertext Markup Language}{Hypertext Markup Language}
\newacr{MIT}{MIT}{MIT}{Massachusetts Institute of Technology}{Massachusetts Institute of Technology}
\newacr{PHP}{PHP}{PHP}{PHP: Hypertext Preprocessor}{PHP: Hypertext Preprocessor}
\newacr{LR}{LR}{LR}{Links nach rechts, rechtsreduzierender Parser}{Links nach rechts, rechtsreduzierender Parser}
\newacr{CSV}{CSV}{CSV}{Comma-Separated Values}{Comma-Separated Values}
\newacr{NT}{NT}{NTs}{New Technology}{New Technologies}
\newacr{SIM}{SIM}{SIM}{Security Information Management}{Security Information Management}
\newacr{SEM}{SEM}{SEM}{Security Event Management}{Security Event Management}
\newacr{EU}{EU}{EU}{Europäische Union}{Europäische Union}
\newacr{DSB}{DSB}{DSB}{Datenschutzbehörde}{Datenschutzbehörden}
\newacr{AMP}{AMP}{AMPs}{Advanced Malware Protection}{Advanced Malware Protections}
\newacr{AP}{AP}{APs}{Access Point}{Access Points}
\newacr{ASA}{ASA}{ASAs}{Adaptive Security Appliance}{Adaptive Security Appliances}
\newacr{ITIL}{ITIL}{ITILs}{\acs{IT} Infrastructure Library}{\acs{IT} Infrastructure Libraries}
\newacr{COBIT}{COBIT}{COBIT}{Control Objectives for Information and Related Technologies}{Control Objectives for Information and Related Technologies}
\newacr{SOX}{SOX}{SOX}{Sarbanes-Oxley Act}{Sarbanes-Oxley Act}
\newacr{ISO}{ISO}{ISO}{Internationale Organisation für Normung}{Internationale Organisation für Normung}
\newacr{IDS}{IDS}{IDSs}{Intrusion Detection System}{Intrusion Detection Systems}
\newacr{AWS}{AWS}{AWS}{Amazon Web Services}{Amazon Web Services}
\newacr{MacOS}{MacOS}{MacOS}{Macintosh Operating System}{Macintosh Operating System}
\newacr{CIM}{CIM}{CIMs}{Common Information Model}{Common Information Models}
\newacr{SPL}{SPL}{SPLs}{Search Processing Language}{Search Processing Languages}
\newacr{VM}{VM}{VMs}{Virtual Machine}{Virtual Machines}
\newacr{CMDB}{CMDB}{CMDBs}{ Configuration Management Database}{ Configuration Management Databases}
\newacr{PC}{PC}{PCs}{Personal Computer}{Personal Computers}


% Als Beschreibung für den Glossareintrag ist der erste Satzt in Wikipedia immer ziemlich hilfreich - glac

\newacrgls{SSID}{SSID}{SSIDs}{Service Set Identifier}{Service Set Identifiers}{ist ein frei wählbarer Name eines Service Sets, durch den es ansprechbar wird. Da diese Kennung oftmals manuell von einem Benutzer in Geräte eingegeben werden muss, ist sie oft eine Zeichenkette, die für Menschen leicht lesbar ist, und sie wird daher allgemein als (Funk-)Netzwerkname des \acsp{WLAN} bezeichnet.}
\newacrgls{SIEM}{SIEM}{SIEMs}{Security Information and Event Management}{Security Information and Event Managements}{kombiniert die zwei Konzepte \acf{SIM} und \acf{SEM} für die Echtzeitanalyse von Sicherheitsalarmen aus den Quellen Anwendungen und Netzwerkkomponenten.}
\newacrgls{LALR}{LALR}{LALR}{Look-ahead \acs{LR} Parser}{Look-ahead \acs{LR} Parser}{Look-ahead \acs{LR} Parser, wobei die Zahl in der Klammer die Anzahl der vorausschauenden Felder beschreibt.}
\newacrgls{RMSProp}{RMSProp}{RMSProp}{Root Mean Square Propagation}{Root Mean Square Propagation}{Ein Optimizer, der die Lernrate für jedes Gewicht dynamisch adaptiert.}
\newacrgls{Adam}{Adam}{Adam}{Adaptive Moment Estimation}{Adaptive Moment Estimation}{Eine neuere Version des \glacs{RMSProp} Optimizers, der performanter ist und auch jedes Gewicht dynamisch adaptiert.}
\newacrgls{LSTM}{LSTM}{LSTM}{Long short-term memory}{Long short-term memories}{Long short-term memory (langes Kurzzeitgedächtnis) ist eine Technik, die dafür sorgt, dass Neurale Netzwerke ein Gedächtnis haben.}

% gl

\newgls{MongoDB}{MongoDB}{MongoDBs}{eine dokumentenorientierte \acs{NoSQL}-\acl{DB}.\newline\href{https://www.mongodb.com/}{https://www.mongodb.com/}}
\newgls{Python}{Python}{Pythons}{eine universelle, üblicherweise interpretierte höhere Programmiersprache.\newline\href{https://www.python.org/}{https://www.python.org/}}
\newgls{Index}{Index}{Indizes}{eine von der Datenstruktur getrennte Indexstruktur in einer Datenbank, die die Suche und das Sortieren nach bestimmten Feldern beschleunigt.}
\newgls{OSI-Modell}{OSI\glsadd{OSI}-Modell}{OSI\glsadd{OSI}-Modelle}{ein Referenzmodell für Netzwerkprotokolle als Schichtenarchitektur.}
\newgls{OSI-Schicht}{OSI\glsadd{OSI}-Schicht}{OSI\glsadd{OSI}-Schichten}{Das \gl{OSI-Modell} hat sieben Schichten: 1 -- Bitübertragung (Physical), 2 -- Sicherung (Data Link), 3 -- Vermittlung-/Paket (Network), 4 -- Transport (Transport), 5 -- Sitzung (Session), 6 -- Darstellung (Presentation), 7 -- Anwendung (Application)}
\newgls{Daemon}{Daemon}{Daemons}{bezeichnet unter Unix oder unixartigen Systemen ein Programm, das im Hintergrund abläuft und bestimmte Dienste zur Verfügung stellt.}
\newgls{Thread}{Thread}{Threads}{bezeichnet einen Ausführungsstrang oder eine Ausführungsreihenfolge in der Abarbeitung eines Programms. Ein Thread ist Teil eines Prozesses.}
\newgls{Socket}{Socket}{Sockets}{ist ein vom Betriebssystem bereitgestelltes Objekt, das als Kommunikationsendpunkt dient. Ein Programm verwendet Sockets, um Daten mit anderen Programmen auszutauschen.}





\newacr{ACL}{ACL}{ACLs}{Access Control List}{Access Control Lists}
\newacr{AD}{AD}{ADs}{Active Directory}{Active Directorys}
\newacr{API}{API}{APIs}{Application Programming Interface}{Application Programming Interfaces}
\newacr{ATA}{ATA}{ATAs}{Advanced Threat Analytics}{Advanced Threat Analytics}
\newacr{DoS}{DoS}{DoS'}{Denial of Service}{Denial of Services}
\newacr{DPI}{DPI}{DPIs}{Deep Packet Inspection}{Deep Packet Inspections}
\newacr{DB}{DB}{DBs}{Datenbank}{Datenbanken}
\newacr{DBMS}{DBMS}{DBMSs}{Datenbankmanagementsystem}{Datenbankmanagementsysteme}
\newacr{DNS}{DNS}{DNS}{Domain Name System}{Domain Name Systeme}
\newacr{DSGVO}{DSGVO}{DSGVOs}{Datenschutz-Grundverordnung}{Datenschutz-Grundverordnungen}
\newacr{DSG}{DSG}{DSGs}{Datenschutzgesetz}{Datenschutzgesetze}
\newacr{EDV}{EDV}{EDVs}{Elektronische Datenverarbeitung}{Elektronische Datenverarbeitungen}
\newacr{FQDN}{FQDN}{FQDNs}{Fully-Qualified Domain Name}{Fully-Qualified Domain Names}
\newacr{GPO}{GPO}{GPOs}{Group Policy Object}{Group Policy Objects}
\newacr{HTL}{HTL}{HTLs}{Höhere Technische Lehranstalt}{Höhere Technische Lehranstalten}
\newacr{HTTP}{HTTP}{HTTP}{Hypertext Transfer Protocol}{Hypertext Transfer Protocol}
\newacr{HTTPS}{HTTPS}{HTTPS}{Hypertext Transfer Protocol Secure}{Hypertext Transfer Protocol Secure}
\newacr{IP}{IP}{IPs}{Internet Protocol}{Internet Protocols}
\newacr{ISP}{ISP}{ISPs}{Internet Service Provider}{Internet Service Providers}
\newacr{ID}{ID}{IDs}{Identifikator}{Identifikatoren}
\newacr{IT}{IT}{ITs}{Informationstechnologie}{Informationstechnologien}
\newacr{JSON}{JSON}{JSONs}{JavaScript Object Notation}{JavaScript Object Notations}
\newacr{KI}{KI}{KIs}{Künstliche Intelligenz}{Künstliche Intelligenzen}
\newacr{KMU}{KMU}{KMUs}{kleines und mittleres Unternehmen}{kleine und mittlere Unternehmen}
\newacr{LAN}{LAN}{LANs}{Local Area Network}{Local Area Networks}
\newacr{MAC}{MAC}{MACs}{Media Access Control}{Media Access Controls}
\newacr{NAT}{NAT}{NATs}{Network Address Translation}{Network Address Translations}
\newacr{NGFW}{NGFW}{NGFWs}{Next Generation Firewall}{Next Generation Firewalls}
\newacr{NoSQL}{NoSQL}{NoSQLs}{Not only \acs{SQL}}{Not only \acsp{SQL}}
\newacr{NPS}{NPS}{NPSs}{Network Policy Server}{Network Policy Server}
\newacr{NTP}{NTP}{NTPs}{Network Time Protocol}{Network Time Protcols}
\newacr{OID}{OID}{OIDs}{Object Identifier}{Object Identifier}
\newacr{OSI}{OSI}{OSIs}{Open Systems Interconnection}{Open Systems Interconnections}
\newacr{OUI}{OUI}{OUIs}{Organizationally Unique Identifier}{Organizationally Unique Identifier}
\newacr{PDU}{PDU}{PDUs}{Protocol Data Unit}{Protocol Data Units}
\newacr{AQL}{AQL}{AQLs}{\textit{Argos-Query-Language}}{\textit{Argos-Query-Languages}}
\newacr{RFC}{RFC}{RFCs}{Request for Comments}{Request for Comments}
\newacr{SaaS}{SaaS}{SaaSs}{Software as a Service}{Software as a Services}
\newacr{SID}{SID}{SID}{Security Identifier}{Security Identifier}
\newacr{SQL}{SQL}{SQLs}{Structured Query Language}{Structured Query Languages}
\newacr{SME}{SME}{SMEs}{small and medium-sized enterprise}{small and medium-sized enterprises}
\newacr{SNMP}{SNMP}{SNMPs}{Simple Network Monitoring Protocol}{Simple Network Monitoring Protocols}
\newacr{SSL}{SSL}{SSLs}{Secure Socket Layer}{Secure Socket Layers}
\newacr{TCP}{TCP}{TCPs}{Transmission Control Protocol}{Transmission Control Protocols}
\newacr{TLS}{TLS}{TLSs}{Transport Layer Security}{Transport Layer Securities}
\newacr{TTL}{TTL}{TTLs}{Time to Live}{Times to Live}
\newacr{UDP}{UDP}{UDPs}{User Datagram Protocol}{User Datagram Protocols}
\newacr{VLAN}{VLAN}{VLANs}{Virtual \acs{LAN}}{Virtual \acsp{LAN}}
\newacr{VPN}{VPN}{VPNs}{Virtual Private Network}{Virtual Private Networks}
\newacr{WLAN}{WLAN}{WLANs}{Wireless \acs{LAN}}{Wireless \acsp{LAN}}
\newacr{WLC}{WLC}{WLCs}{\acs{WLAN}-Controller}{\acs{WLAN}-Controller}
\newacr{XML}{XML}{XMLs}{Extensible Markup Language}{Extensible Markup Languages}
\newacr{IPFIX}{IPFIX}{IPFIXs}{\acs{IP} Flow Information Export}{\acs{IP} Flow Information Exports}
\newacr{MIB}{MIB}{MIBs}{Management Information Base}{Management Information Bases}
\newacr{MO}{MO}{MOs}{Managed Object}{Managed Objects}
\newacr{ASN.1}{ASN.1}{ASN.1}{Abstract Syntax Notation One}{Abstract Syntax Notation One}
\newacr{ASCII}{ASCII}{ASCII}{American Standard Code for Information Interchange}{American Standard Code for Information Interchange}
\newacr{RAM}{RAM}{RAM}{Random Access Memory}{Random Access Memory}
\newacr{CPU}{CPU}{CPUs}{Central Processing Unit}{Central Processing Units}
\newacr{CLI}{CLI}{CLIs}{Command Line Interface}{Command Line Interfaces}
\newacr{LWAP}{LWAP}{LWAPs}{Light Weight \ac{AP}}{Light Weight \acp{AP}}
\newacr{VTY}{VTY}{VTYs}{Virtual Terminal Line}{Virtual Terminal Lines}
\newacr{SMTP}{SMTP}{SMTP}{Simple Mail Transfer Protocol}{Simple Mail Transfer Protocol}
\newacr{GUI}{GUI}{GUIs}{Graphical User Interface}{Graphical User Interfaces}
\newacr{IPS}{IPS}{IPSs}{Intrusion Prevention System}{Intrusion Prevention Systems}
\newacr{UI}{UI}{UIs}{User Interface}{User Interfaces}
\newacr{CSS}{CSS}{CSS}{Cascading Style Sheets}{Cascading Style Sheets}
\newacr{HTML}{HTML}{HTML}{Hypertext Markup Language}{Hypertext Markup Language}
\newacr{MIT}{MIT}{MIT}{Massachusetts Institute of Technology}{Massachusetts Institute of Technology}
\newacr{PHP}{PHP}{PHP}{PHP: Hypertext Preprocessor}{PHP: Hypertext Preprocessor}
\newacr{LR}{LR}{LR}{Links nach rechts, rechtsreduzierender Parser}{Links nach rechts, rechtsreduzierender Parser}
\newacr{CSV}{CSV}{CSV}{Comma-Separated Values}{Comma-Separated Values}
\newacr{NT}{NT}{NTs}{New Technology}{New Technologies}
\newacr{SIM}{SIM}{SIM}{Security Information Management}{Security Information Management}
\newacr{SEM}{SEM}{SEM}{Security Event Management}{Security Event Management}
\newacr{EU}{EU}{EU}{Europäische Union}{Europäische Union}
\newacr{DSB}{DSB}{DSB}{Datenschutzbehörde}{Datenschutzbehörden}
\newacr{AMP}{AMP}{AMPs}{Advanced Malware Protection}{Advanced Malware Protections}
\newacr{AP}{AP}{APs}{Access Point}{Access Points}
\newacr{ASA}{ASA}{ASAs}{Adaptive Security Appliance}{Adaptive Security Appliances}
\newacr{ITIL}{ITIL}{ITILs}{\acs{IT} Infrastructure Library}{\acs{IT} Infrastructure Libraries}
\newacr{COBIT}{COBIT}{COBIT}{Control Objectives for Information and Related Technologies}{Control Objectives for Information and Related Technologies}
\newacr{SOX}{SOX}{SOX}{Sarbanes-Oxley Act}{Sarbanes-Oxley Act}
\newacr{ISO}{ISO}{ISO}{Internationale Organisation für Normung}{Internationale Organisation für Normung}
\newacr{IDS}{IDS}{IDSs}{Intrusion Detection System}{Intrusion Detection Systems}
\newacr{AWS}{AWS}{AWS}{Amazon Web Services}{Amazon Web Services}
\newacr{MacOS}{MacOS}{MacOS}{Macintosh Operating System}{Macintosh Operating System}
\newacr{CIM}{CIM}{CIMs}{Common Information Model}{Common Information Models}
\newacr{SPL}{SPL}{SPLs}{Search Processing Language}{Search Processing Languages}
\newacr{VM}{VM}{VMs}{Virtual Machine}{Virtual Machines}
\newacr{CMDB}{CMDB}{CMDBs}{ Configuration Management Database}{ Configuration Management Databases}
\newacr{PC}{PC}{PCs}{Personal Computer}{Personal Computers}


% Als Beschreibung für den Glossareintrag ist der erste Satzt in Wikipedia immer ziemlich hilfreich - glac

\newacrgls{SSID}{SSID}{SSIDs}{Service Set Identifier}{Service Set Identifiers}{ist ein frei wählbarer Name eines Service Sets, durch den es ansprechbar wird. Da diese Kennung oftmals manuell von einem Benutzer in Geräte eingegeben werden muss, ist sie oft eine Zeichenkette, die für Menschen leicht lesbar ist, und sie wird daher allgemein als (Funk-)Netzwerkname des \acsp{WLAN} bezeichnet.}
\newacrgls{SIEM}{SIEM}{SIEMs}{Security Information and Event Management}{Security Information and Event Managements}{kombiniert die zwei Konzepte \acf{SIM} und \acf{SEM} für die Echtzeitanalyse von Sicherheitsalarmen aus den Quellen Anwendungen und Netzwerkkomponenten.}
\newacrgls{LALR}{LALR}{LALR}{Look-ahead \acs{LR} Parser}{Look-ahead \acs{LR} Parser}{Look-ahead \acs{LR} Parser, wobei die Zahl in der Klammer die Anzahl der vorausschauenden Felder beschreibt.}
\newacrgls{RMSProp}{RMSProp}{RMSProp}{Root Mean Square Propagation}{Root Mean Square Propagation}{Ein Optimizer, der die Lernrate für jedes Gewicht dynamisch adaptiert.}
\newacrgls{Adam}{Adam}{Adam}{Adaptive Moment Estimation}{Adaptive Moment Estimation}{Eine neuere Version des \glacs{RMSProp} Optimizers, der performanter ist und auch jedes Gewicht dynamisch adaptiert.}
\newacrgls{LSTM}{LSTM}{LSTM}{Long short-term memory}{Long short-term memories}{Long short-term memory (langes Kurzzeitgedächtnis) ist eine Technik, die dafür sorgt, dass Neurale Netzwerke ein Gedächtnis haben.}

% gl

\newgls{MongoDB}{MongoDB}{MongoDBs}{eine dokumentenorientierte \acs{NoSQL}-\acl{DB}.\newline\href{https://www.mongodb.com/}{https://www.mongodb.com/}}
\newgls{Python}{Python}{Pythons}{eine universelle, üblicherweise interpretierte höhere Programmiersprache.\newline\href{https://www.python.org/}{https://www.python.org/}}
\newgls{Index}{Index}{Indizes}{eine von der Datenstruktur getrennte Indexstruktur in einer Datenbank, die die Suche und das Sortieren nach bestimmten Feldern beschleunigt.}
\newgls{OSI-Modell}{OSI\glsadd{OSI}-Modell}{OSI\glsadd{OSI}-Modelle}{ein Referenzmodell für Netzwerkprotokolle als Schichtenarchitektur.}
\newgls{OSI-Schicht}{OSI\glsadd{OSI}-Schicht}{OSI\glsadd{OSI}-Schichten}{Das \gl{OSI-Modell} hat sieben Schichten: 1 -- Bitübertragung (Physical), 2 -- Sicherung (Data Link), 3 -- Vermittlung-/Paket (Network), 4 -- Transport (Transport), 5 -- Sitzung (Session), 6 -- Darstellung (Presentation), 7 -- Anwendung (Application)}
\newgls{Daemon}{Daemon}{Daemons}{bezeichnet unter Unix oder unixartigen Systemen ein Programm, das im Hintergrund abläuft und bestimmte Dienste zur Verfügung stellt.}
\newgls{Thread}{Thread}{Threads}{bezeichnet einen Ausführungsstrang oder eine Ausführungsreihenfolge in der Abarbeitung eines Programms. Ein Thread ist Teil eines Prozesses.}
\newgls{Socket}{Socket}{Sockets}{ist ein vom Betriebssystem bereitgestelltes Objekt, das als Kommunikationsendpunkt dient. Ein Programm verwendet Sockets, um Daten mit anderen Programmen auszutauschen.}


% für Titelseite
% https://www.reddit.com/r/LaTeX/comments/sfyusz/vertical_line_before_all_bullets_in_itemize/
\usepackage{tcolorbox}
\tcbuselibrary{xparse,skins,breakable}
%% Farbe rgb: 255,51,0
\definecolor{htl3red}{RGB}{255,51,0}
\newtcolorbox{TitlePageBox}{%
    breakable,
    blanker,
    left=1em,
    borderline west={0.15cm}{3pt}{htl3red},
}

%% sollte das letzte Package sein
\usepackage[unicode=true,
 bookmarks=true,bookmarksnumbered=false,bookmarksopen=false,
 breaklinks=true,pdfborder={0 0 0},backref=false,colorlinks=false]
 {hyperref}
\hypersetup{pdftitle={Diplomarbeit Titel},
 pdfauthor={Wer auch immer},
 pdfsubject={Diplomarbeit},
 pdfkeywords={dies, das}}
\urlstyle{same} % don't use monospace font for urls

% \usepackage{cleveref}  % optional, noch bessere Querverweise

% \usepackage{showkeys} %% DEBUG, Label anzeigen

%% for pandoc
\providecommand{\tightlist}{%
  \setlength{\itemsep}{0pt}\setlength{\parskip}{0pt}}

%% braucht man manchmal -- wenn er über passthrough mault -- TODO: warum?
%\newcommand{\passthrough}[1]{\lstset{mathescape=false}\texttt{#1}\lstset{mathescape=true}}

% Auch Fußnoten bündig ausrichten
\deffootnote[]{1em}{1em}{\textsuperscript{\thefootnotemark\ }}
%% setup
\sloppy % weniger Meldungen
\voffset7mm % etwas nach unten

%%%%%%%%%%%%%%%%%%%%%%%%%%%%%%%%%%%%%%%%%%%%%%%%%%%%%%%%%%%%%%%%%%%%%%%%%%%%%%%%%%
\begin{document}
%% schöner: 10000 -- gar keine, 1000 als Mittelweg
\clubpenalty = 1000 % Schusterjungen verhindern
\widowpenalty = 1000 % Hurenkinder verhindern
\displaywidowpenalty = 1000

%% wir schreiben keine Umlaut mit "a "o
\shorthandoff{"}
%% mit kapitelautor kann man den Autor festlegen oder auf leer setzen - steht dann in der Fußzeile.
%% bitte immer (gleich) nach der Überschrift setzen, nicht vorher -- sonst steht es bei Kapiteln eventuell eine Seite zu früh
\newcommand{\kapitelautor}{}

%%
\newcommand{\strong}[1]{\textbf{#1}}
\newcommand{\code}[1]{\texttt{#1}}

% einfaches "siehe ..." - das Ziel muss man markieren mit \label{name} -- macht pandoc automatisch
% einfache Variante
%\newcommand{\kap}[1]{Kapitel~\ref{#1}, Seite~\pageref{#1}}
%\newcommand{\siehe}[1]{siehe \kap{#1}}
%\newcommand{\abb}[1]{Abbildung~\ref{#1}, Seite~\pageref{#1}}
% bessere Variante - braucht varioref
\newcommand{\kap}[1]{Kapitel~\vref{#1}}
\newcommand{\siehe}[1]{siehe \kap{#1}}
\newcommand{\abb}[1]{Abbildung~\vref{#1}}


%% http://ieg.ifs.tuwien.ac.at/~aigner/download/tuwien.sty
%Div. Abkürzungen (in Anlehnung an Jochen Köpper, jkthesis):
%\RequirePackage{xspace}
\newcommand{\bzw}{bzw.\@\xspace}
\newcommand{\bzgl}{bzgl.\@\xspace}
\newcommand{\ca}{ca.\@\xspace}
\newcommand{\dah}{d.\thinspace{}h.\@\xspace}
\newcommand{\Dah}{D.\thinspace{}h.\@\xspace}
\newcommand{\ds}{d.\thinspace{}s.\@\xspace}
\newcommand{\evtl}{evtl.\@\xspace}
\newcommand{\ua}{u.\thinspace{}a.\@\xspace}
\newcommand{\Ua}{U.\thinspace{}a.\@\xspace}
\newcommand{\usw}{usw.\@\xspace}
\newcommand{\va}{v.\thinspace{}a.\@\xspace}
\newcommand{\vgl}{vgl.\@\xspace}
\newcommand{\zB}{z.\thinspace{}B.\@\xspace}
\newcommand{\ZB}{Zum Beispiel\xspace}

%% https://github.com/Digital-Media/HagenbergThesis
\newcommand{\latex}{La\-TeX\xspace} % kein schnoerkeliges LaTeX mehr
\newcommand{\tex}{TeX\xspace}       % kein schnoerkeliges TeX mehr
\newcommand{\bs}{\textbackslash}    % Backslash character
\newcommand{\obnh}{\hskip 0pt } %optional break without hyphen: e.g. PlugIn{\obnh}Filter

\newcommand{\sa}{s.\ auch\@\xspace}
\newcommand{\so}{s.\ oben\xspace}
\newcommand{\su}{s.\ unten\@\xspace}

\newcommand{\uae}{u.\thinspace{}\"A.\@\xspace}
\newcommand{\uva}{u.\thinspace{}v.\thinspace{}a.\@\xspace}
\newcommand{\uvm}{u.\thinspace{}v.\thinspace{}m.\@\xspace}



%%%%%Anfang Titelseite
%\pagenumbering{roman}

%% Farbe rgb: 255,51,0 Strich + neuer Text

\frontmatter % Switches to roman numbering
\title{Diplomarbeit}

\begin{titlepage}
\begin{minipage}[b]{1\columnwidth}
\parbox[b]{99mm}{
\begin{TitlePageBox}
\footnotesize% klein
\textsf{% und ohne Rifen
\textbf{\textsc{Höhere Technische Bundeslehranstalt} Wien 3, Rennweg}\\
\\
Höhere Abteilung für Mechatronik\\
Höhere Abteilung für Informationstechnologie\\
Fachschule für Informationstechnik}
\end{TitlePageBox}
}\hfill\parbox[b]{50mm}{\includegraphics[width=51mm]{HTL3RLogoRGB}}
\mbox{}
\end{minipage}

\vspace{1cm}

\begin{center}
\textbf{\LARGE Diplomarbeit}\\
\vspace{15mm}
\textbf{\large{}\todo{TODO}eventuell KURZTITEL}\\
\textbf{\large{}Ausgeschriebener Titel der Diplomarbeit}\\

\vfill
ausgeführt an der\\
Höheren Abteilung für Informationstechnologie/??Ausbildungsschwerpunkt??\\
der Höheren Technischen Lehranstalt Wien 3 Rennweg\\

\vfill
im Schuljahr 20??/20??\\


\vspace{1cm}
{
\renewcommand{\arraystretch}{1.8}
\begin{tabular}{l c r}
durch  & \hfill & unter Anleitung von \\
\textbf{\large{}Mitarbeiter:in Eins (alphabetisch)} && Individualbetreuer:in Eins \\
\textbf{\large{}Mitarbeiter:in Zwei} && Individualbetreuer:in Zwei \\
\textbf{\large{}Mitarbeiter:in Drei} && Individualbetreuer:in Drei \\
\textbf{\large{}Mitarbeiter:in Vier} && Individualbetreuer:in Vier \\
\end{tabular}
}

\vfill

Wien, \today
\par\end{center}

\end{titlepage}%%%%%%%%%%%%%%%%%%%%% Ende Titelseite %%%%%%%%%%%%%%%%%%%%%%
% \renewcommand{\kapitelautor}{}  % bleibt leer (allg. Teil)
\chapter*{Kurzfassung}


Worum geht es?

Wie wurde vorgegangen?

Was wurde herausgefunden?

Was bedeuten deine Ergebnisse?


\blindtext[1]


\chapter*{Abstract}
\selectlanguage{english}

Thats why -- the translated text ,,Kurzfassung`` (this should be
a translation).

Im englischen Abstract sollte inhaltlich das Gleiche stehen wie in
der deutschen Kurzfassung. Versuchen Sie daher, die Kurzfassung präzise
umzusetzen, ohne aber dabei Wort für Wort zu übersetzen. Beachten
Sie bei der Übersetzung, dass gewisse Redewendungen aus dem Deutschen
im Englischen kein Pendant haben oder völlig anders formuliert werden
müssen und dass die Satzstellung im Englischen sich (bekanntlich)
vom Deutschen stark unterscheidet. Es empfiehlt sich übrigens – auch
bei höchstem Vertrauen in die persönlichen Englischkenntnisse – eine
kundige Person für das „proof reading“ zu engagieren. Die richtige
Übersetzung für „Diplomarbeit“ ist übrigens schlicht thesis, allenfalls
„diploma thesis“ oder „Master’s thesis“, auf keinen Fall aber „diploma
work“ oder gar „dissertation“\citep{hagenberg}.

Wichtig ist wegen des Abteilens ein \code{\textbackslash{}begin\{english\}}
bzw. \code{\textbackslash{}selectlanguage\{ngerman\}}.

\selectlanguage{ngerman}

\chapter*{Ehrenwörtliche Erklärung}

Ich erkläre an Eides statt, dass ich die individuelle Themenstellung
selbstständig und ohne fremde Hilfe verfasst, keine anderen als die
angegebenen Quellen und Hilfsmittel benutzt und die den benutzten
Quellen wörtlich und inhaltlich entnommenen Stellen als solche erkenntlich
gemacht habe.

\begin{flushleft}
\bigskip{}
Wien, am \today \\
\newcommand{\namesigdate}[2][8cm]{
\vspace{2cm}~\newline
\parbox{#1}{\hrule\centering #2\Large\strut}
\hfill
}
\namesigdate{Mitarbeiter:in Eins}
\namesigdate{Mitarbeiter:in Zwei}
\namesigdate{Mitarbeiter:in Drei}
\par\end{flushleft}



%%%%%%%%%%%%%%%%%%%%%%%%%%%%%%%%%%%%%%%%%%%%%%%%%%%%%%%%%%%%%%%%%%%%%%%%%%%%%%%%%%%%%%%%
\cleardoublepage{}
\tableofcontents{}
\cleardoublepage{}
\listoftables
\todo{kann entfallen falls (fast) leer}
\cleardoublepage{}
\listoffigures


%hier geht es los mit dem Text - auf einer rechten Seite
\cleardoublepage{}
%\pagenumbering{arabic}
\mainmatter

\chapter{Ziele}
% \renewcommand{\kapitelautor}{}  % bleibt eventuell leer (gemeinsame Arbeit)

Das erste Kapitel stellt die Ziele der DA (inkl. individuelle Ziele
aller Mitarbeiter) dar.\todo{viel Text schreiben}

Mögliche Gliederung (nach \cite{leitfaden})

\begin{itemize}
\item  Einleitung
\item   Zielsetzung und Aufgabenstellung des Gesamtprojekts
\item   individuelle Zielsetzung und Aufgabenstellung mit Terminplan der einzelnen Teammitglieder
\item   Grundlagen und Methoden (Ist-Situation, Lösungsansätze, konkrete Vorgehensweise)
\item   Bearbeitung der Aufgabenstellung (technische Beschreibungen, Berechnungen)
\item   Ergebnisse (Ergebnisdarstellung, kritische Gegenüberstellung mit der Zielsetzung
 und der gewählten Vorgehensweise)
\end{itemize}

Mögliche Variante: Ziele laut Antrag, mit Querverweisen zu den einzelnen Kapiteln.

\chapter{Formatierung}

% wer hat diese Kapitel geschrieben oder leer
\renewcommand{\kapitelautor}{Autor: Hans Huber}


\section{Vorlagen}

In diesen Kapitel gibt es einige Muster für Dinge die oft vorkommen.

\subsection{Abweichungen}

Die in diesem Dokument getroffenen Einstellungen bzw. das resultierende PDF ist die
Referenz für die Diplomarbeiten der Abteilung IT.

%Abweichungen sind nur in begründeten Ausnahmefällen und nach Rücksprache mit dem Betreuer zulässig.

\subsection{Formatvorlagen}

Alle Formatierungen sollten mit Formatvorlagen vorgenommen werden.
Spätestens bei der Konvertierung in ein Ebook rächen sich diese \quotedblbase Sünden``:
Ebooks sind HTML Dokumente mit einer Formatierung mittels CSS.

Auch bei der Umwandlung in interaktive PDFs ist eine konsequente Formatierung
wichtig.


\subsection{Schriften und Absätze}

Hier findet man eine Beschreibung des Layouts -- Details folgen weiter
unten.
\begin{description}
\item [{Schrift:}] dieses \LaTeX{}-Dokument verwendet die Standardschriften.
Die Schriftgröße soll 12\,pt betragen.
\item [{Absatz:}] entweder verwendet man wie in \LaTeX{} einen etwas größeren
Seitenrand oder einen größeren Zeilenabstand. Beides sorgt für bessere
Lesbarkeit. Zwischen den Abätzen ist ein Abstand. Alternative: die
erste Zeile eines Absatzes wird etwas eingerückt (nicht die erste
Zeile nach einer Überschrift, nach einem Bild etc.) und bzw. oder
es gibt einen Abstand zwischen den Absätzen. Am Ende und Anfang einer
Seite sollten mindestens zwei Zeilen eines Absatzes sein (keine Schusterjungen\footnote{siehe \url{http://www.typolexikon.de/s/schusterjunge.html}}
und Hurenkinder\footnote{siehe \url{http://www.typolexikon.de/h/hurenkind.html}}).
\item [{Blocksatz:}] Alle Texte werden im Blocksatz gesetzt. Die Silbentrennung
ist dann obligatorisch.
\item [{Kapitelüberschriften:}] Überschriften erster Ordnung sollten auf
rechten Seiten beginnen. Über jeder Überschrift sollte ein Abstand
sein. Alle Überschriften müssen mit de nächsten Absatz \quotedblbase zusammengehalten``
werden -- keine einsamen Überschriften am Ende einer Seite.
\item [{Inhaltsverzeichnis:}] das Inhaltsverzeichnis sollte möglichst kompakt
sein. Als Gliederung dienen fette Hauptüberschriften und etwas Abstand
über den Zeilen.
\item [{Seitenformat:}] der Ausdruck erfolgt zweiseitig, ein entsprechender
Bundsteg ist zu berücksichtigen\footnote{Die Einstellung der Seitenränder ist keinesfalls beliebig. Sie sollte
bewährten Regeln folgen, {[}\ldots{}{]}. Die häufige Zielvorgabe
\quotedblbase Den Platz auf dem Papier möglichst gut ausnutzen``
ist keine typografische sondern eine extrem laienhafte Regel. aus
\citep{layout}}. Nach Rücksprache mit dem Betreuer kann auch eine einseitige Variante
gewählt werden. Bei Bedarf könne auch einzelne Seiten im Querformat
gesetzt werden.
\item [{Kopfzeile:}] die Kopfzeile sollte dieser Vorlage entsprechen. Falls,
nach Rücksprache mit dem Betreuer, der Ausdruck nur in Schwarz-weiß
erfolgt, kann das Logo entfallen.
\item [{Fußzeile:}] hier ist Platz für den Autor des Kapitels und die Seitennummer.
Wie bei technischen Publikationen üblich ist die Einleitung und die
Verzeichnisse mit römischen Seitennummern versehen. Das eigentliche
Dokument wird mit arabischen Ziffern nummeriert. Beide Nummerierungen
sind unabhängig voneinander und beginnen jeweils bei 1.
\item [{Autor:}] Jedes Kapitel muss auch einem Autor haben. Das sieht man
in der Fußzeile oder als Textbox in der Nähe der Überschrift. Alternativ
kann es im Anhang eine Liste geben. Das ist besonders wichtig wenn
es viele Beilagen, z.B. Handbücher ohne direkte Angabe des Autors,
gibt.
\item [{PDF:}] Die PDF Metainformation sollten richtig sein (Autor etc.)
-- siehe Datei/Eigenschaften. Links auf Webseiten, Verweise innerhalb
des Dokuments, das Inhaltsverzeichnis, die Fußnoten usw. sollten \quotedblbase klickbar``
sein.
\end{description}

\subsection{Bilder\label{sub:Bilder}}

Das Bild als Gleitobjekt ist genau hier, oder oben auf der Seite,
oder unten, aber immer zentriert mit Nummer und Beschreibung -- wenn
es sinnvoll ist auch mit Querverweis (siehe Abbildung \ref{Bild11}).
Durch Gleitobjekte, d.~h. Bilder oben oder unten auf der Seite statt
\quotedblbase genau hier``, werden halbleere Seiten durch besonders
große Bilder vermieden.

Wichtig: alle Bilder oder andere Medien z.~B. Screenshots, Audio
oder Video für EBooks und interaktive PDFs sollten mit einen entsprechenden
Quellennachweis versehen sein.

\begin{figure}[tbh]
\begin{centering}
\includegraphics[width=5cm]{HTL3RLogoRGB}
\par\end{centering}

\caption{Ein Bild}
\label{Bild11}
\end{figure}

Eingefügte Screenshots sind im Ausdruck meistens unscharf. Ursache
\begin{itemize}
\item zu niedrige Auflösung
\item als JPEG mit verlustbehafteter Komprimierung gespeichert
\end{itemize}
Abhilfe: z.B. \url{https://meyerweb.com/eric/thoughts/2018/08/24/firefoxs-screenshot-command-2018/}
%% shift-F2 screenshot --dpr 4

\subsection{Tabellen}

In der folgenden Tabelle sieht man: es gibt immer eine Nummer und
eine Beschreibung. Besonders längere Tabellen sollten eventuell als
Gleitobjekt am Ende oder Anfang einer Seite positioniert werden. Geht
die Tabelle über mehrere Seiten so ist die Überschrift zu wiederholen.

\begin{table}[h]
\begin{centering}
\begin{tabular}{|c|c|c|}
\hline
Überschrift & Wert & noch einer\tabularnewline
\hline
\hline
1 & abc & Hallo\tabularnewline
\hline
2 & def & Latex\tabularnewline
\hline
\end{tabular}
\par\end{centering}

\caption{So eine tolle Tabelle}
\end{table}



\subsection{Formel}

Etwas Text als Überleitung zu einer Formel:

\[
f(x)=\left\{ \begin{array}{cc}
\log_{8}x & x>0\\
0 & x=0\\
\sum_{i=1}^{5}\alpha_{i}+\sqrt{-\frac{1}{x}} & x<0
\end{array}\right.
\]


Wenn man sehr viele Formeln hat sollte man diese auch nummerieren.
Besonders bei Verweisen ist das sehr sinnvoll.

Hinweis: auch für Werte wie \SI{100}{\mebi\byte} gibt es ein eigenes Paket -- siunitx.


\subsection{Sourcecode}


Wichtig: keine endlos langen Listings ohne Erklärung. Besser sind kurze, im Text erläuterte Ausschnitte
des Codes, ohne Code-Kommentare. Man kann \zB auch die Fehlerbehandlung entfernen und stattdessen auf
den Quelltext verweisen.

Sourcecode sollte in einer Schrift mit fixer Breite sein. Der Zeilenabstand sollte möglichst gering sein.
Falls man Verweise braucht sollte man die Listings auch nummerieren.

% das kann auch ganz oben stehen
% das braucht man nur einmal
\lstset{numbers=left, numberstyle=\tiny, stepnumber=2, numbersep=5pt, showspaces=true, frame=single}
% einmal oder immer was anderes
\lstset{language=C}

% hier könnte man auch aus Dateien lesen
\begin{lstlisting}
#include <stdio.h>

int main()
{
  printf("Hello world\n");
}
\end{lstlisting}

Die genaue Formatierung ist freigestellt: Einstellungen wie bunt bzw.
fett, Markierung von Leerzeichen und Zeilennummerierung kann an den
Bedarf der Diplomarbeit angepasst werden.

Beispiel Java mit anderen Einstellungen -- nur als Beispiel, in der
Diplomarbeit sollte man sich an eine einheitliches Format halten.
Bei längeren Listings muss man eventuell mit Umbrüchen rechnen, oder
man verwendet einen Rahmen der frei angeordnet werden kann (\siehe{sub:Bilder}).

% Einstellungen für die fogenden Listings
% entweder mit \begin{listing} oder in Lyx als Programmlisting
\lstset{numbers=right, numberstyle=\tiny, stepnumber=2, numbersep=5pt, showspaces=false, frame=single}
\lstset{language=Java}

Achtung \LaTeX{}-User: Listing kann keine Umlaute, aber unter \citep{listingtipp}
gibt es eine Lösung.

\begin{lstlisting}[caption={Java Beispiel},captionpos=b]
import java.awt.*;
import java.awt.event.*;

public class AL extends Frame
                 implements WindowListener, ActionListener {
  TextField text = new TextField(20);
  Button b;
  private int numClicks = 0;

  public static void main(String[] args) {
    AL myWindow = new AL("My first window");
    myWindow.setSize(350,100);
    myWindow.setVisible(true);
  }
}
\end{lstlisting}

\needspace{2cm}
Hinweise:
\begin{itemize}
\item Pandoc erzeugt automatisch bunte Listings
\item und es gibt die Pakete listings und minted
\item aber man sollte die drei Varianten nicht mischen!
\end{itemize}


\subsection{Fachbegriffe}

Fachbegriffe in einer Fremdsprache oder Kommandos sollten einheitlich
gekennzeichnet werden. Bei Latex verwendet man dazu \quotedblbase logisches
Markup``, bei Word oder Open/Libre-office wird all diesen Wörtern
wird eine Vorlage zugewiesen, das Aussehen wird dann an einer Stelle
zentral festgelegt.

Als Beispiel soll \emph{Text to Speech}\index{Text to Speech: Umwandlung von Texten in Sprache}
dienen. Solche Wörter sollte natürlich in ein Glossar aufgenommen
werden.

Oder der Befehl \strong{dir} für die Kommandozeile. Die Angabe von
Dateinamen sollte auch einheitlich sein: entweder \emph{/etc/passwd}
oder \strong{C:\textbackslash{}system32}.


\subsection{Zitieren}

Die Quellenangabe kann in Form eines Vollbelegs in der Fußnote\footnote{aus Zitat --- Wikipedia, Die freie Enzyklopädie, \url{http://de.wikipedia.org/w/index.php?title=Zitat},
Abgerufen 2014-09-14}(bei technischen Dokumenten eher unüblich) oder als Kurzbeleg am Schluss
der gesamten Arbeit aufgeführt werden. Beim Kurzbeleg sind dabei verschiedene
Formen üblich. Der platzsparendste, aber am wenigsten aussagekräftige
Zitierstil ist die fortlaufende Nummerierung aller zitierten Quellen
{[}123{]}.


Insbesondere in der Informatik üblich ist eine Kombination der ersten drei Buchstaben
des Autorennamens und, soweit vorhanden, der letzten beiden Ziffern des Erscheinungsjahres
(z. B. „The04“ für Theisen 2004). Diese Variante wird auch in dieser Vorlage verwendet.
Siehe \cite{zitate}.

Alternative: \verb+\footcite{zitate}+, braucht andere Aufrufe zum Bauen.

Wohl am weitesten verbreitet im nicht technischen Bereich ist
der vollständige Verfassernamen mit Erscheinungsjahr, wobei mehrere
Quellen desselben Autors innerhalb eines Jahres durch fortlaufende
Buchstaben kenntlich gemacht werden (z. B. „Theisen 2004c“). Weniger
üblich, aber am aussagekräftigsten ist die Quellenangabe unter Hinzufügung
eines Schlagwortes, das den mit der Materie vertrauten Leser zumeist
bereits die zitierte Quelle erkennen lässt, z. B. in der Form „Theisen
(Wissenschaftliches Arbeiten, 2004)“.

Obwohl mehrere Zitierstile bzw. Zitiertechniken zur Verfügung stehen,
werden in einem Dokument üblicherweise nicht mehrere verwendet; ein
ausgewählter Zitierstil wird im gesamten Dokument konsequent beibehalten.
Ein gute Übersicht bietet \citep{wiki:zitat}.

Auch unter \url{http://www.diplomarbeiten-bbs.at/zitation-plagiate}
gibt es gute Informationen zum Thema.


\subsubsection{Quellenverzeichnis}

Unter \LaTeX{} kann das Programm Bib\TeX{} zur Erstellung von Literaturangaben
verwendet werden.
\begin{itemize}
\item Links auf Wikipedia sollten vermieden werden.\nopagebreak
\item Jeder Link sollte mit einem Abfragedatum versehen sein.\nopagebreak
\item Das Literaturverzeichnis kommt an das Ende des Dokuments.\nopagebreak
\end{itemize}
Viele Details dazu findet man bei \citep{wiki:zitat}.


\subsubsection{Rechtliches zum Zitieren}

Achtung: nicht gekennzeichnete Zitate (Plagiate) führen zu einer negativen
Beurteilung der Diplomarbeit. Nach \citep{wiki:quelle}:

§ 57 des österreichischen Urheberrechtsgesetzes\citep{ris57} enthält
detaillierte Vorschriften über die Quellenangabe, unter anderem: Werden
Stellen oder Teile von Sprachwerken nach §\,46 vervielfältigt, so
sind sie in der Quellenangabe so genau zu bezeichnen, dass sie in
dem benutzten Werke leicht aufgefunden werden können. In den Erläuterungen
(ErlRV) heißt es: Bei Entlehnungen aus umfangreichen Werken muss also
in der Quellenangabe auch die Seite, der Abschnitt, das Kapitel oder
der Akt, wo sich die entlehnte Stelle befindet, angeführt werden (Dillenz,
Materialien zum österreichischen Urheberrecht, 134, zitiert nach \citep{dittrich},
S. 621)

2002 nahm der österreichische OGH zur Frage der Quellenangabe in der
Entscheidung Riven Rock Stellung: Nach § 57 Abs 4 UrhG bedarf die
Unterlassung einer Quellenangabe der Rechtfertigung durch die im redlichen
Verkehr geltenden Gewohnheiten und Gebräuche. Bei Auslegung dieser
Bestimmung ist eine Abwägung der Interessen des Urhebers mit jenen
des zur freien Werknutzung Berechtigten nach dem Verständnis loyaler,
den Belangen des Urhebers mit Verständnis gegenübertretenden, billig
und gerecht denkenden Benutzern (Vinck aaO § 63 Rz 2) geboten und
danach zu beurteilen, ob dem freien Werknutzer neben der Nennung des
Autors/Verlags auch die Nennung des Namens des Übersetzers von in
einer Rundfunksendung verlesenen Roman-Zitaten zumutbar ist.


\section{Inhalt}


\subsection{Aussagen}

Alles im Buch sollte man in folgende drei Typen einteilen:
\begin{enumerate}
\item Stand der Technik -- Quelle notwendig
\item selbst gemacht -- Verweis wie und wo
\item Meinung -- als eigene Meinung bzw. Einschätzung kennzeichnen
\end{enumerate}

\subsection{Bad Practice}

Was man vermeiden sollte -- diese Dinge führen zur Mehrarbeit und
verursachen zusätzlichen Stress in der hektischen Zeit knapp vor dem
Abgabetermin.


\paragraph{Stil}
\begin{itemize}
\item Extrem lange, geschachtelte Sätze und/oder endlose Textpassagen ohne
Gliederung durch Absätze.\nopagebreak

\begin{itemize}
\item Vielleicht bzw. sinnvollerweise lassen Sie den Text auch von einer
\quotedblbase außenstehenden Person`` lesen.
\end{itemize}
\item Aufzählungen im Text statt Listen. Wie man hier sieht dürfen bei Listen
auch mehrere Sätze stehen.
\item \uline{Unterstreichen} ist ein Relikt aus \quotedblbase Schreibmaschinen-Zeiten``.
\item Eine Diplomarbeit ist keine Erzählung. Natürlich kann man als \quotedblbase ich``
oder \quotedblbase wir`` auf \quotedblbase unsere Probleme`` eingehen,
aber im Allgemeinen ist ein formaler, beschreibender und technischer
Stil einzuhalten.
\item Eine Diplomarbeit ist auch keine Email oder SMS: Schreiben Sie ganze
Sätze ohne kryptische Abkürzungen und Smileys.
\item Ein Mindestmaß an Interpunktion wird vorausgesetzt. Eventuell lassen
Sie den Text durch eine kundige Person Ihres Vertrauens korrigieren.
\item Es gibt viele verschiedene Striche, und alle sehen verschieden aus:
Gedankenstriche, Bindestriche und Minus kommen in einer Diplomarbeit
häufig vor.
\item Weitere Wörter die Ihren Betreuer verzweifeln lassen -- natürlich
nur bei übermäßiger Verwendung

\begin{itemize}
\item Welcher/Welches, Hierbei
\end{itemize}
\end{itemize}

\paragraph{Technik}
\begin{itemize}
\item (viele) händische Formatierungen statt Formatvorlagen.
\item zusätzliche manuelle Seitenumbrüche oder Leerzeilen für ein \quotedblbase schöneres``
Layout. Es gibt bei den Absatzformatierungen tolle Möglichkeiten für
Abstände vor und nach einem Absatz bzw. zum Beeinflussen des Textflusses.
Mittels \zB \code{\textbackslash{}needspace\{2cm\}} kann man für
\textit{genügend} Platz sorgen.
\item Arbeiten Sie mit dem Programm statt gegen das Programm:

\begin{itemize}
\item Verweise als fixer Text. Nutzen Sie die Möglichkeiten der Textverarbeitung.
\item Dinge die \quotedblbase kompliziert`` einzugeben sind, sind meist
falsch -- richtige Lösungen sind in allen Programmen auch \quotedblbase leicht``
zu erreichen\footnote{Oder Sie verwenden ein für Ihre Zwecke schlecht geeignetes Programm.}.
\end{itemize}
\item Kontrollieren Sie beim fertigen PDF die Angaben unter Datei / Eigenschaften
-- dort sollten sinnvolle Dinge stehen. Bei Latex wird dazu das Paket
\texttt{\code{\texttt{hyperref}}} verwendet.
\end{itemize}

\section{Details zu Formatierung}


\subsection{Schriftarten}

Die Word- und Libreoffice-vorlage verwenden etwas andere Schriften
als das Latex Dokument.


\section{Beispiele}


\subsection{Zitieren mit Latex}

Am Beispiel der URL \url{http://de.wikibooks.org/wiki/LaTeX-Kompendium}
und des Buches \quotedblbase \LaTeX{}: Einführung``.

Man braucht eine \texttt{.bib} Datei mit den notwendigen Informationen:

\begin{lstlisting}[language={[LaTeX]TeX}]
@book{kopka1991latex,
  title={LaTeX: Einfuehrung},
  author={Kopka, Helmut and Rahtz, Sebastian},
  volume={2},
  year={1991},
  publisher={Addison-Wesley}
}

@online{latexKomp,
  author = {},
  title ={LaTeX-Kompendium - Wikibooks, Sammlung freier Lehr-,
Sach- und Fachbuecher},
  url = {http://de.wikibooks.org/wiki/LaTeX-Kompendium},
  lastchecked = {2014.09.14},
  key={LaTeX-Kompendium}
}
\end{lstlisting}


Diese Einträge werden dann im Text verwendet: \texttt{\textbackslash{}cite\{kopka1991latex\}
}und das Ergebnis sieht dann so \citep{kopka1991latex} und so \citep{latexKomp}
aus. Gleichzeitig erscheinen diese Einträge auch im Literaturverzeichnis
am Ende des Dokuments.


\subsection{Direkte Formatierungen -- sollte man vermeiden}

Der Autor kann natürlich auch eingreifen und zum Beispiel\\
Zeilenumbrüche erzwingen und \\
\\
Leerzeilen -- aber das sollte man nicht machen. \newpage{}

Ein Seitenumbruch kostet mich auch nicht mehr als einen müden Lacher,
ist aber noch seltener wirklich sinnvoll.
Besser ist \texttt{\textbackslash{}needspace\{5cm\}} -- neue Seite, aber nur wenn weniger als 5~cm Platz übrig sind.

Schriftgrößen ändern:
\begin{itemize}
\item {\tiny{}tiny}
\item {\scriptsize{}scriptsize}
\item {\footnotesize{}footnotesize}
\item {\small{}small}
\item normalsize
\item {\large{}large}
\item {\Large{}Large}
\item {\LARGE{}LARGE}
\item {\huge{}huge}
\item {\Huge{}Huge}
\end{itemize}

\uline{ein Strich unten drunter} oder \uuline{sogar zwei}.




\chapter{Planung}

% Das komplette nächste Kapitel wird in der externen Datei diplomarbeit2.tex gespeichert.
% Es wird an dieser Stelle im Dokument eingebaut.
% Damit ist es möglich, mehrere Personen an diversen Teilen der Diplomarbeit arbeiten zu lassen.

\input{markdown/diplomarbeit2.md.tex}

\input{markdown/einstellungen.md.tex}


\chapter{Umsetzung-Dummy}

% wer hat diese Kapitel geschrieben oder leer
\renewcommand{\kapitelautor}{Autor: Blindtext}

\Blindtext


\chapter{Ergebnisse-Dummy}

% wer hat diese Kapitel geschrieben oder leer
\renewcommand{\kapitelautor}{Autor: Susi Sorglos}

\blindmathpaper\Blindtext


\chapter{Evaluation-Dummy}

% wer hat diese Kapitel geschrieben oder leer 
\renewcommand{\kapitelautor}{Autor: Blindtext}

\Blindtext

\Blinddocument\Blindtext\Blinddocument[2]\Blindtext\Blinddocument[5]\Blindtext\Blinddocument[10]\Blindtext\Blindtext

%%%%%%%%%%%%%%%%%%%%%%%%%%%%%%%%%%%%%%%%%%%%%%%%%%%%%%%%%%%%%%%%%%%%%%%%%%%%%%%%%%%%%%%%%%
% wer hat diese Kapitel geschrieben oder leer
\renewcommand{\kapitelautor}{}



\appendix

\chapter{Anhang 1\label{chap:Anhang-1}}

was auch immer: technische Dokumentationen etc.

Zusätzlich sollte es geben:
\begin{itemize}
\item Abkürzungsverzeichnis
\item Quellenverzeichnis (hier: Bibtex im Stil plaindin)
\item optional: Akronyme und Glossar
\end{itemize}

%% optional: Akronyme und Glossar
% kann man löschen falls kein Glossar gebraucht
\printglossary[type=\acronymtype, title=Abkürzungsverzeichnis, toctitle=Abkürzungsverzeichnis]
\printglossary[type=main, title=Glossar, toctitle=Glossar]

\printindex{}

%% Flattersatz -- damit werden die langen URLs besser umgebrochen
\raggedright %% eventuell auskommentieren
%\bibliographystyle{plaindin}%Alternative unsrtdin - Nummern im Text aufsteigend
\bibliographystyle{alphadin}
\bibliography{diplom}


\cleardoublepage
\newcommand{\Messbox}[2]{%Parameters: #1=Breite, #2=Hoehe
\setlength{\unitlength}{1.0mm}%
\begin{picture}(#1,#2)%
\linethickness{0.05mm}%
\put(0,0){\dashbox{0.2}(#1,#2)%
{\parbox{#1mm}{%
\centering\footnotesize
%{\bf MESSBOX}\\
% if \textrm fails use \rm
Breite $ = #1 {\textrm\ mm}$\\
Höhe $ = #2 {\textrm\ mm}$
}}}\end{picture}
}
\begin{center} {\Large --- Druckgröße kontrollieren! ---}
\bigskip

\Messbox{100}{50} % Angabe der Breite/Hoehe in mm
\bigskip

{\Large --- Diese Seite nach dem (Probe-)Druck entfernen! ---}
\todo{Diese Seite nach dem  (Probe-)Druck entfernen!
Nicht notwendig wenn Druck+Binden extern passiert.}
\end{center}
\end{document}
