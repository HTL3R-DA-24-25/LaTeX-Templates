
\section{Vorlagen}

In diesen Kapitel gibt es einige Muster für Dinge die oft vorkommen.

\subsection{Abweichungen}

Die in diesem Dokument getroffenen Einstellungen bzw. das resultierende PDF ist die
Referenz für die Diplomarbeiten der Abteilung IT.

%Abweichungen sind nur in begründeten Ausnahmefällen und nach Rücksprache mit dem Betreuer zulässig.

\subsection{Formatvorlagen}

Alle Formatierungen sollten mit Formatvorlagen vorgenommen werden.
Spätestens bei der Konvertierung in ein Ebook rächen sich diese \quotedblbase Sünden``:
Ebooks sind HTML Dokumente mit einer Formatierung mittels CSS.

Auch bei der Umwandlung in interaktive PDFs ist eine konsequente Formatierung
wichtig.


\subsection{Schriften und Absätze}

Hier findet man eine Beschreibung des Layouts -- Details folgen weiter
unten.
\begin{description}
\item [{Schrift:}] dieses \LaTeX{}-Dokument verwendet die Standardschriften.
Die Schriftgröße soll 12\,pt betragen.
\item [{Absatz:}] entweder verwendet man wie in \LaTeX{} einen etwas größeren
Seitenrand oder einen größeren Zeilenabstand. Beides sorgt für bessere
Lesbarkeit. Zwischen den Abätzen ist ein Abstand. Alternative: die
erste Zeile eines Absatzes wird etwas eingerückt (nicht die erste
Zeile nach einer Überschrift, nach einem Bild etc.) und bzw. oder
es gibt einen Abstand zwischen den Absätzen. Am Ende und Anfang einer
Seite sollten mindestens zwei Zeilen eines Absatzes sein (keine Schusterjungen\footnote{siehe \url{http://www.typolexikon.de/s/schusterjunge.html}}
und Hurenkinder\footnote{siehe \url{http://www.typolexikon.de/h/hurenkind.html}}).
\item [{Blocksatz:}] Alle Texte werden im Blocksatz gesetzt. Die Silbentrennung
ist dann obligatorisch.
\item [{Kapitelüberschriften:}] Überschriften erster Ordnung sollten auf
rechten Seiten beginnen. Über jeder Überschrift sollte ein Abstand
sein. Alle Überschriften müssen mit de nächsten Absatz \quotedblbase zusammengehalten``
werden -- keine einsamen Überschriften am Ende einer Seite.
\item [{Inhaltsverzeichnis:}] das Inhaltsverzeichnis sollte möglichst kompakt
sein. Als Gliederung dienen fette Hauptüberschriften und etwas Abstand
über den Zeilen.
\item [{Seitenformat:}] der Ausdruck erfolgt zweiseitig, ein entsprechender
Bundsteg ist zu berücksichtigen\footnote{Die Einstellung der Seitenränder ist keinesfalls beliebig. Sie sollte
bewährten Regeln folgen, {[}\ldots{}{]}. Die häufige Zielvorgabe
\quotedblbase Den Platz auf dem Papier möglichst gut ausnutzen``
ist keine typografische sondern eine extrem laienhafte Regel. aus
\citep{layout}}. Nach Rücksprache mit dem Betreuer kann auch eine einseitige Variante
gewählt werden. Bei Bedarf könne auch einzelne Seiten im Querformat
gesetzt werden.
\item [{Kopfzeile:}] die Kopfzeile sollte dieser Vorlage entsprechen. Falls,
nach Rücksprache mit dem Betreuer, der Ausdruck nur in Schwarz-weiß
erfolgt, kann das Logo entfallen.
\item [{Fußzeile:}] hier ist Platz für den Autor des Kapitels und die Seitennummer.
Wie bei technischen Publikationen üblich ist die Einleitung und die
Verzeichnisse mit römischen Seitennummern versehen. Das eigentliche
Dokument wird mit arabischen Ziffern nummeriert. Beide Nummerierungen
sind unabhängig voneinander und beginnen jeweils bei 1.
\item [{Autor:}] Jedes Kapitel muss auch einem Autor haben. Das sieht man
in der Fußzeile oder als Textbox in der Nähe der Überschrift. Alternativ
kann es im Anhang eine Liste geben. Das ist besonders wichtig wenn
es viele Beilagen, z.B. Handbücher ohne direkte Angabe des Autors,
gibt.
\item [{PDF:}] Die PDF Metainformation sollten richtig sein (Autor etc.)
-- siehe Datei/Eigenschaften. Links auf Webseiten, Verweise innerhalb
des Dokuments, das Inhaltsverzeichnis, die Fußnoten usw. sollten \quotedblbase klickbar``
sein.
\end{description}

\subsection{Bilder\label{sub:Bilder}}

Das Bild als Gleitobjekt ist genau hier, oder oben auf der Seite,
oder unten, aber immer zentriert mit Nummer und Beschreibung -- wenn
es sinnvoll ist auch mit Querverweis (siehe Abbildung \ref{Bild11}).
Durch Gleitobjekte, d.~h. Bilder oben oder unten auf der Seite statt
\quotedblbase genau hier``, werden halbleere Seiten durch besonders
große Bilder vermieden.

Wichtig: alle Bilder oder andere Medien z.~B. Screenshots, Audio
oder Video für EBooks und interaktive PDFs sollten mit einen entsprechenden
Quellennachweis versehen sein.

\begin{figure}[tbh]
\begin{centering}
\includegraphics[width=5cm]{HTL3RLogoRGB}
\par\end{centering}

\caption{Ein Bild}
\label{Bild11}
\end{figure}

Eingefügte Screenshots sind im Ausdruck meistens unscharf. Ursache
\begin{itemize}
\item zu niedrige Auflösung
\item als JPEG mit verlustbehafteter Komprimierung gespeichert
\end{itemize}
Abhilfe: z.B. \url{https://meyerweb.com/eric/thoughts/2018/08/24/firefoxs-screenshot-command-2018/}
%% shift-F2 screenshot --dpr 4

\subsection{Tabellen}

In der folgenden Tabelle sieht man: es gibt immer eine Nummer und
eine Beschreibung. Besonders längere Tabellen sollten eventuell als
Gleitobjekt am Ende oder Anfang einer Seite positioniert werden. Geht
die Tabelle über mehrere Seiten so ist die Überschrift zu wiederholen.

\begin{table}[h]
\begin{centering}
\begin{tabular}{|c|c|c|}
\hline
Überschrift & Wert & noch einer\tabularnewline
\hline
\hline
1 & abc & Hallo\tabularnewline
\hline
2 & def & Latex\tabularnewline
\hline
\end{tabular}
\par\end{centering}

\caption{So eine tolle Tabelle}
\end{table}



\subsection{Formel}

Etwas Text als Überleitung zu einer Formel:

\[
f(x)=\left\{ \begin{array}{cc}
\log_{8}x & x>0\\
0 & x=0\\
\sum_{i=1}^{5}\alpha_{i}+\sqrt{-\frac{1}{x}} & x<0
\end{array}\right.
\]


Wenn man sehr viele Formeln hat sollte man diese auch nummerieren.
Besonders bei Verweisen ist das sehr sinnvoll.

Hinweis: auch für Werte wie \SI{100}{\mebi\byte} gibt es ein eigenes Paket -- siunitx.


\subsection{Sourcecode}


Wichtig: keine endlos langen Listings ohne Erklärung. Besser sind kurze, im Text erläuterte Ausschnitte
des Codes, ohne Code-Kommentare. Man kann \zB auch die Fehlerbehandlung entfernen und stattdessen auf
den Quelltext verweisen.

Sourcecode sollte in einer Schrift mit fixer Breite sein. Der Zeilenabstand sollte möglichst gering sein.
Falls man Verweise braucht sollte man die Listings auch nummerieren.

% das kann auch ganz oben stehen
% das braucht man nur einmal
\lstset{numbers=left, numberstyle=\tiny, stepnumber=2, numbersep=5pt, showspaces=true, frame=single}
% einmal oder immer was anderes
\lstset{language=C}

% hier könnte man auch aus Dateien lesen
\begin{lstlisting}
#include <stdio.h>

int main()
{
  printf("Hello world\n");
}
\end{lstlisting}

Die genaue Formatierung ist freigestellt: Einstellungen wie bunt bzw.
fett, Markierung von Leerzeichen und Zeilennummerierung kann an den
Bedarf der Diplomarbeit angepasst werden.

Beispiel Java mit anderen Einstellungen -- nur als Beispiel, in der
Diplomarbeit sollte man sich an eine einheitliches Format halten.
Bei längeren Listings muss man eventuell mit Umbrüchen rechnen, oder
man verwendet einen Rahmen der frei angeordnet werden kann (\siehe{sub:Bilder}).

% Einstellungen für die fogenden Listings
% entweder mit \begin{listing} oder in Lyx als Programmlisting
\lstset{numbers=right, numberstyle=\tiny, stepnumber=2, numbersep=5pt, showspaces=false, frame=single}
\lstset{language=Java}

Achtung \LaTeX{}-User: Listing kann keine Umlaute, aber unter \citep{listingtipp}
gibt es eine Lösung.

\begin{lstlisting}[caption={Java Beispiel},captionpos=b]
import java.awt.*;
import java.awt.event.*;

public class AL extends Frame
                 implements WindowListener, ActionListener {
  TextField text = new TextField(20);
  Button b;
  private int numClicks = 0;

  public static void main(String[] args) {
    AL myWindow = new AL("My first window");
    myWindow.setSize(350,100);
    myWindow.setVisible(true);
  }
}
\end{lstlisting}

\needspace{2cm}
Hinweise:
\begin{itemize}
\item Pandoc erzeugt automatisch bunte Listings
\item und es gibt die Pakete listings und minted
\item aber man sollte die drei Varianten nicht mischen!
\end{itemize}


\subsection{Fachbegriffe}

Fachbegriffe in einer Fremdsprache oder Kommandos sollten einheitlich
gekennzeichnet werden. Bei Latex verwendet man dazu \quotedblbase logisches
Markup``, bei Word oder Open/Libre-office wird all diesen Wörtern
wird eine Vorlage zugewiesen, das Aussehen wird dann an einer Stelle
zentral festgelegt.

Als Beispiel soll \emph{Text to Speech}\index{Text to Speech: Umwandlung von Texten in Sprache}
dienen. Solche Wörter sollte natürlich in ein Glossar aufgenommen
werden.

Oder der Befehl \strong{dir} für die Kommandozeile. Die Angabe von
Dateinamen sollte auch einheitlich sein: entweder \emph{/etc/passwd}
oder \strong{C:\textbackslash{}system32}.


\subsection{Zitieren}

Die Quellenangabe kann in Form eines Vollbelegs in der Fußnote\footnote{aus Zitat --- Wikipedia, Die freie Enzyklopädie, \url{http://de.wikipedia.org/w/index.php?title=Zitat},
Abgerufen 2014-09-14}(bei technischen Dokumenten eher unüblich) oder als Kurzbeleg am Schluss
der gesamten Arbeit aufgeführt werden. Beim Kurzbeleg sind dabei verschiedene
Formen üblich. Der platzsparendste, aber am wenigsten aussagekräftige
Zitierstil ist die fortlaufende Nummerierung aller zitierten Quellen
{[}123{]}.


Insbesondere in der Informatik üblich ist eine Kombination der ersten drei Buchstaben
des Autorennamens und, soweit vorhanden, der letzten beiden Ziffern des Erscheinungsjahres
(z. B. „The04“ für Theisen 2004). Diese Variante wird auch in dieser Vorlage verwendet.
Siehe \cite{zitate}.

Alternative: \verb+\footcite{zitate}+, braucht andere Aufrufe zum Bauen.

Wohl am weitesten verbreitet im nicht technischen Bereich ist
der vollständige Verfassernamen mit Erscheinungsjahr, wobei mehrere
Quellen desselben Autors innerhalb eines Jahres durch fortlaufende
Buchstaben kenntlich gemacht werden (z. B. „Theisen 2004c“). Weniger
üblich, aber am aussagekräftigsten ist die Quellenangabe unter Hinzufügung
eines Schlagwortes, das den mit der Materie vertrauten Leser zumeist
bereits die zitierte Quelle erkennen lässt, z. B. in der Form „Theisen
(Wissenschaftliches Arbeiten, 2004)“.

Obwohl mehrere Zitierstile bzw. Zitiertechniken zur Verfügung stehen,
werden in einem Dokument üblicherweise nicht mehrere verwendet; ein
ausgewählter Zitierstil wird im gesamten Dokument konsequent beibehalten.
Ein gute Übersicht bietet \citep{wiki:zitat}.

Auch unter \url{http://www.diplomarbeiten-bbs.at/zitation-plagiate}
gibt es gute Informationen zum Thema.


\subsubsection{Quellenverzeichnis}

Unter \LaTeX{} kann das Programm Bib\TeX{} zur Erstellung von Literaturangaben
verwendet werden.
\begin{itemize}
\item Links auf Wikipedia sollten vermieden werden.\nopagebreak
\item Jeder Link sollte mit einem Abfragedatum versehen sein.\nopagebreak
\item Das Literaturverzeichnis kommt an das Ende des Dokuments.\nopagebreak
\end{itemize}
Viele Details dazu findet man bei \citep{wiki:zitat}.


\subsubsection{Rechtliches zum Zitieren}

Achtung: nicht gekennzeichnete Zitate (Plagiate) führen zu einer negativen
Beurteilung der Diplomarbeit. Nach \citep{wiki:quelle}:

§ 57 des österreichischen Urheberrechtsgesetzes\citep{ris57} enthält
detaillierte Vorschriften über die Quellenangabe, unter anderem: Werden
Stellen oder Teile von Sprachwerken nach §\,46 vervielfältigt, so
sind sie in der Quellenangabe so genau zu bezeichnen, dass sie in
dem benutzten Werke leicht aufgefunden werden können. In den Erläuterungen
(ErlRV) heißt es: Bei Entlehnungen aus umfangreichen Werken muss also
in der Quellenangabe auch die Seite, der Abschnitt, das Kapitel oder
der Akt, wo sich die entlehnte Stelle befindet, angeführt werden (Dillenz,
Materialien zum österreichischen Urheberrecht, 134, zitiert nach \citep{dittrich},
S. 621)

2002 nahm der österreichische OGH zur Frage der Quellenangabe in der
Entscheidung Riven Rock Stellung: Nach § 57 Abs 4 UrhG bedarf die
Unterlassung einer Quellenangabe der Rechtfertigung durch die im redlichen
Verkehr geltenden Gewohnheiten und Gebräuche. Bei Auslegung dieser
Bestimmung ist eine Abwägung der Interessen des Urhebers mit jenen
des zur freien Werknutzung Berechtigten nach dem Verständnis loyaler,
den Belangen des Urhebers mit Verständnis gegenübertretenden, billig
und gerecht denkenden Benutzern (Vinck aaO § 63 Rz 2) geboten und
danach zu beurteilen, ob dem freien Werknutzer neben der Nennung des
Autors/Verlags auch die Nennung des Namens des Übersetzers von in
einer Rundfunksendung verlesenen Roman-Zitaten zumutbar ist.


\section{Inhalt}


\subsection{Aussagen}

Alles im Buch sollte man in folgende drei Typen einteilen:
\begin{enumerate}
\item Stand der Technik -- Quelle notwendig
\item selbst gemacht -- Verweis wie und wo
\item Meinung -- als eigene Meinung bzw. Einschätzung kennzeichnen
\end{enumerate}

\subsection{Bad Practice}

Was man vermeiden sollte -- diese Dinge führen zur Mehrarbeit und
verursachen zusätzlichen Stress in der hektischen Zeit knapp vor dem
Abgabetermin.


\paragraph{Stil}
\begin{itemize}
\item Extrem lange, geschachtelte Sätze und/oder endlose Textpassagen ohne
Gliederung durch Absätze.\nopagebreak

\begin{itemize}
\item Vielleicht bzw. sinnvollerweise lassen Sie den Text auch von einer
\quotedblbase außenstehenden Person`` lesen.
\end{itemize}
\item Aufzählungen im Text statt Listen. Wie man hier sieht dürfen bei Listen
auch mehrere Sätze stehen.
\item \uline{Unterstreichen} ist ein Relikt aus \quotedblbase Schreibmaschinen-Zeiten``.
\item Eine Diplomarbeit ist keine Erzählung. Natürlich kann man als \quotedblbase ich``
oder \quotedblbase wir`` auf \quotedblbase unsere Probleme`` eingehen,
aber im Allgemeinen ist ein formaler, beschreibender und technischer
Stil einzuhalten.
\item Eine Diplomarbeit ist auch keine Email oder SMS: Schreiben Sie ganze
Sätze ohne kryptische Abkürzungen und Smileys.
\item Ein Mindestmaß an Interpunktion wird vorausgesetzt. Eventuell lassen
Sie den Text durch eine kundige Person Ihres Vertrauens korrigieren.
\item Es gibt viele verschiedene Striche, und alle sehen verschieden aus:
Gedankenstriche, Bindestriche und Minus kommen in einer Diplomarbeit
häufig vor.
\item Weitere Wörter die Ihren Betreuer verzweifeln lassen -- natürlich
nur bei übermäßiger Verwendung

\begin{itemize}
\item Welcher/Welches, Hierbei
\end{itemize}
\end{itemize}

\paragraph{Technik}
\begin{itemize}
\item (viele) händische Formatierungen statt Formatvorlagen.
\item zusätzliche manuelle Seitenumbrüche oder Leerzeilen für ein \quotedblbase schöneres``
Layout. Es gibt bei den Absatzformatierungen tolle Möglichkeiten für
Abstände vor und nach einem Absatz bzw. zum Beeinflussen des Textflusses.
Mittels \zB \code{\textbackslash{}needspace\{2cm\}} kann man für
\textit{genügend} Platz sorgen.
\item Arbeiten Sie mit dem Programm statt gegen das Programm:

\begin{itemize}
\item Verweise als fixer Text. Nutzen Sie die Möglichkeiten der Textverarbeitung.
\item Dinge die \quotedblbase kompliziert`` einzugeben sind, sind meist
falsch -- richtige Lösungen sind in allen Programmen auch \quotedblbase leicht``
zu erreichen\footnote{Oder Sie verwenden ein für Ihre Zwecke schlecht geeignetes Programm.}.
\end{itemize}
\item Kontrollieren Sie beim fertigen PDF die Angaben unter Datei / Eigenschaften
-- dort sollten sinnvolle Dinge stehen. Bei Latex wird dazu das Paket
\texttt{\code{\texttt{hyperref}}} verwendet.
\end{itemize}

\section{Details zu Formatierung}


\subsection{Schriftarten}

Die Word- und Libreoffice-vorlage verwenden etwas andere Schriften
als das Latex Dokument.


\section{Beispiele}


\subsection{Zitieren mit Latex}

Am Beispiel der URL \url{http://de.wikibooks.org/wiki/LaTeX-Kompendium}
und des Buches \quotedblbase \LaTeX{}: Einführung``.

Man braucht eine \texttt{.bib} Datei mit den notwendigen Informationen:

\begin{lstlisting}[language={[LaTeX]TeX}]
@book{kopka1991latex,
  title={LaTeX: Einfuehrung},
  author={Kopka, Helmut and Rahtz, Sebastian},
  volume={2},
  year={1991},
  publisher={Addison-Wesley}
}

@online{latexKomp,
  author = {},
  title ={LaTeX-Kompendium - Wikibooks, Sammlung freier Lehr-,
Sach- und Fachbuecher},
  url = {http://de.wikibooks.org/wiki/LaTeX-Kompendium},
  lastchecked = {2014.09.14},
  key={LaTeX-Kompendium}
}
\end{lstlisting}


Diese Einträge werden dann im Text verwendet: \texttt{\textbackslash{}cite\{kopka1991latex\}
}und das Ergebnis sieht dann so \citep{kopka1991latex} und so \citep{latexKomp}
aus. Gleichzeitig erscheinen diese Einträge auch im Literaturverzeichnis
am Ende des Dokuments.


\subsection{Direkte Formatierungen -- sollte man vermeiden}

Der Autor kann natürlich auch eingreifen und zum Beispiel\\
Zeilenumbrüche erzwingen und \\
\\
Leerzeilen -- aber das sollte man nicht machen. \newpage{}

Ein Seitenumbruch kostet mich auch nicht mehr als einen müden Lacher,
ist aber noch seltener wirklich sinnvoll.
Besser ist \texttt{\textbackslash{}needspace\{5cm\}} -- neue Seite, aber nur wenn weniger als 5~cm Platz übrig sind.

Schriftgrößen ändern:
\begin{itemize}
\item {\tiny{}tiny}
\item {\scriptsize{}scriptsize}
\item {\footnotesize{}footnotesize}
\item {\small{}small}
\item normalsize
\item {\large{}large}
\item {\Large{}Large}
\item {\LARGE{}LARGE}
\item {\huge{}huge}
\item {\Huge{}Huge}
\end{itemize}

\uline{ein Strich unten drunter} oder \uuline{sogar zwei}.

